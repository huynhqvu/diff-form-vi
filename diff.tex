%% LyX 2.1.4 created this file.  For more info, see http://www.lyx.org/.
%% Do not edit unless you really know what you are doing.
\documentclass[11pt]{amsbook}
\usepackage{amstext}
\usepackage{amsthm}
\usepackage{mathptmx}
\usepackage{newtxmath}
\usepackage[T1]{fontenc}
\usepackage[utf8]{inputenc}
\usepackage[a4paper]{geometry}
\geometry{verbose,tmargin=2cm,bmargin=2cm,lmargin=2.5cm,rmargin=2.5cm}
\usepackage[vietnam]{babel}
\usepackage{mathtools}
\makeindex
\usepackage{esint}
\usepackage[all]{xy}
\usepackage[unicode=true,
 bookmarks=false,
 breaklinks=false,pdfborder={0 0 1},backref=section,colorlinks=false]
 {hyperref}
\hypersetup{
 pdftex}

\makeatletter
%%%%%%%%%%%%%%%%%%%%%%%%%%%%%% Textclass specific LaTeX commands.
\numberwithin{section}{chapter}
\numberwithin{equation}{section}
\numberwithin{figure}{section}
  \theoremstyle{definition}
  \newtheorem*{example*}{\protect\examplename}
\theoremstyle{plain}
\ifx\thechapter\undefined
\newtheorem{thm}{\protect\theoremname}
\else
\newtheorem{thm}{\protect\theoremname}[chapter]
\fi
  \theoremstyle{remark}
  \newtheorem{rem}[thm]{\protect\remarkname}
  \theoremstyle{plain}
  \newtheorem{prop}[thm]{\protect\propositionname}
  \theoremstyle{definition}
  \newtheorem{example}[thm]{\protect\examplename}
  \theoremstyle{remark}
  \newtheorem*{rem*}{\protect\remarkname}
  \theoremstyle{plain}
  \newtheorem{cor}[thm]{\protect\corollaryname}
  \theoremstyle{definition}
  \newtheorem{defn}[thm]{\protect\definitionname}
  \theoremstyle{plain}
  \newtheorem*{thm*}{\protect\theoremname}
  \theoremstyle{definition}
  \newtheorem{problem}[thm]{\protect\problemname}

%%%%%%%%%%%%%%%%%%%%%%%%%%%%%% User specified LaTeX commands.
\usepackage[vietnamese]{babel}
\usepackage[all]{xy}

\usepackage{graphicx}
\usepackage{pgf}
\usepackage{tikz}

\usepackage{pgfplots}
\usetikzlibrary{arrows}

\newcommand{\RR}{\mathbb R} 

\def\firstofone#1{#1}\let\uppercase\firstofone\let\MakeUppercase\firstofone

\swapnumbers

 \theoremstyle{theorem}
\newtheorem{theorem}[equation]{Định lí}
 %same counter for equation, theorem
\makeatletter
    \let\c@equation\c@figure  %same counter for equation, figure
\makeatother

\numberwithin{equation}{section}
\numberwithin{figure}{section}


\renewenvironment{thm}{\begin{theorem}}{\end{theorem}}
\newtheorem*{theorem*}{Định lí}
\renewenvironment{thm*}{\begin{theorem*}}{\end{theorem*}}
%\newtheorem*{theorem*}{Định lí}
%\renewenvironment{thm*}{\begin{theorem*}}{\end{theorem*}}
%\newtheorem{lemma}[theorem]{Bổ đề}
%\renewenvironment{lem}{\begin{lemma}}{\end{lemma}}
%\newtheorem*{lemma*}{Bổ đề}
%\renewenvironment{lem*}{\begin{lemma*}}{\end{lemma*}}

\newtheorem{proposition}[equation]{Mệnh đề}
\renewenvironment{prop}{\begin{proposition}}{\end{proposition}}
%\newtheorem*{proposition*}{Mệnh đề}
%\renewenvironment{prop*}{\begin{proposition*}}{\end{proposition*}}

\newtheorem{corollary}[equation]{Hệ quả}
\renewenvironment{cor}{\begin{corollary}}{\end{corollary}}
%\newtheorem*{corollary*}{Hệ quả}
%\renewenvironment{cor*}{\begin{corollary*}}{\end{corollary*}}

\theoremstyle{definition}
\newtheorem{definition}[equation]{Định nghĩa}
\renewenvironment{defn}{\begin{definition}}{\end{definition}}
%\newtheorem*{definition*}{Định nghĩa}
%\renewenvironment{defn*}{\begin{definition*}}{\end{definition*}}

\theoremstyle{remark}
\newtheorem{remark}[equation]{Ghi chú}
\renewenvironment{rem}{\begin{remark}}{\end{remark}}
\newtheorem*{remark*}{Ghi chú}
\renewenvironment{rem*}{\begin{remark*}}{\end{remark*}}
\newtheorem{ex}[equation]{Ví dụ}
\renewenvironment{example}{\begin{ex}}{\end{ex}}
\newtheorem*{ex*}{Ví dụ}
\renewenvironment{example*}{\begin{ex*}}{\end{ex*}}
\newtheorem{prob}[equation]{}
\renewenvironment{problem}{\smaller\begin{prob}}{\end{prob}\normalsize}

\DeclareTextFontCommand{\emph}{\itshape\color{violet}}


\renewcommand{\boxed}[1]{  \fcolorbox{gray}{lime}{$\normalcolor\displaystyle #1$}}

\makeatother

  \providecommand{\corollaryname}{Corollary}
  \providecommand{\definitionname}{Definition}
  \providecommand{\examplename}{Example}
  \providecommand{\problemname}{Problem}
  \providecommand{\propositionname}{Proposition}
  \providecommand{\remarkname}{Remark}
  \providecommand{\theoremname}{Theorem}
\providecommand{\theoremname}{Theorem}

\begin{document}

\title{Bài giảng Dạng vi phân}
\begin{abstract}
Đây là tập bài giảng cho môn cao học Giải tích trên đa tạp, dựa trên
bài giảng của R. Sjamaar \cite{Sjamaar06}. Tập bài giảng này sẽ được
tiếp tục sửa chữa và bổ sung. 

Biên soạn:
\begin{itemize}
\item Phan Đình Hiếu. Học viên cao học Toán Giải tích khóa 2014.{\small \par}
\item Lê Chiêu Hoàng Nguyên. Sinh viên ngành Toán khóa 2012-2016.{\small \par}
\item Phan Văn Phương. Học viên cao học Toán Giải tích khóa 2012.{\small \par}
\item Huỳnh Quang Vũ: người biên tập. Khoa Toán-Tin học, Đại học Khoa học
Tự nhiên, Đại học Quốc gia Thành phố Hồ Chí Minh, email: \texttt{\href{mailto:hqvu@hcmus.edu.vn}{hqvu@hcmus.edu.vn}}{\small \par}
\end{itemize}
11/2015 -- \today
\end{abstract}

\maketitle
\tableofcontents{}


\chapter*{Mở đầu}

Ta nhớ lại trong giải tích cổ điển hàm nhiều biến. Xét tích phân như
${\displaystyle \iint_{D}x^{2}y^{3}dxdy={\displaystyle \iint_{D}x^{2}y^{3}dA}.}$
Ở đây $dA$ là \textquotedbl{}phần tử diện tích\textquotedbl{} được
hiểu một cách mơ hồ mà không định nghĩa rõ. Hay ${\displaystyle \int_{\gamma}xdy+ydx.}$
Hay ${\displaystyle \int_{\gamma}fds}$, với $ds$ là phần tử chiều
dài. Hay ${\displaystyle \iint_{S}fdS}$ với $dS$ là phần tử diện
tích mặt. 

\begin{tikzpicture} \small \draw   (0,0) to [out=30,in=210] (1,1) to [out=30,in=210] (2,2.5) to [out=30,in=240] (4,3);  \node [above] at (1,1) {$\gamma$};   \end{tikzpicture} 

Trong chương trình đại học các đại lượng như vậy không được giải thích
rõ ràng. Môn học này nghiên cứu đối tượng này một cách có hệ thống.
Một mục đích nữa của môn học là phát triển lên đối tượng tổng quát
hơn. Cần hiểu rõ hơn đối tượng là cái gì? Nhiều chiều thì $ds$, $dS$
là cái gì? Các đối tượng trong không gian $\mathbb{R}^{3}$ sẽ được
phát triển lên không gian nhiều chiều.

Chỉ riêng về vi phân hoặc riêng về tích phân đã phát triển trong tổng
quát khá xa, tuy nhiên sự kết hợp cả hai thì còn hạn chế. Xét bài
toán tích phân đơn giản: 
\[
{\displaystyle \int_{0}^{1}xdx=\left.\left({\displaystyle \frac{x^{2}}{2}}\right)\right|_{0}^{1}=\dfrac{1}{2}1^{2}-\dfrac{1}{2}0^{2}}.
\]
 Nhắc lại công thức Newton-Leibniz: Nếu $F'=f$ thì 
\[
\int_{a}^{b}fdx=F|_{a}^{b}.
\]
 Như vậy đạo hàm và tích phân có thể xem là hai phép toán ngược nhau.
Tổng quát hóa của công thức Newton-Leibniz là công thức Stokes sẽ
có trong môn học này.

Tại sao phải xét không gian nhiều chiều? Vì khái niệm không gian tổng
quát không chỉ mô tả không gian ba chiều chúng ta sống mà nó còn mô
tả nhiều hệ thống. Số tham số quy định số chiều của hệ thống. Như
vậy không gian nhiều chiều phụ thuộc nhiều tham số có thể lên đến
vô hạn. Nói cách khác số chiều bằng số bậc tự do của hệ thống, bằng
số tham số độc lập miêu tả hệ thống.
\begin{example*}
Hệ gồm hai quả lắc (pendulum): Với $O$ cố định, cái thứ hai gắn vào
cái thứ nhất.

\begin{tikzpicture}
\small

\draw (-6.46,4.)-- (-4.,4.);
\draw (-6.,6.)-- (-6.,3.62);
\draw (-8.,3.)-- (-5.68,4.16);
\node [left] at (-5.76,6.18) {$\mathbb{R}^3$};

\draw [fill=black] (-4.,5.) circle (2pt);
\node [above] at (-4.,5.) {$O$};
\draw [fill=lime] (-3.,3.) circle (1.5pt);
\node [right] at (-3.,3.) {$x$};
\draw [fill=lime] (-5.,2.) circle (1.5pt);
\node [left]  at (-5.,2.) {$y$};
\draw (-4.,5.)-- (-3.,3.);
\draw (-3.,3.)-- (-5.,2.);

\end{tikzpicture}
\end{example*}
Bài toán định vị trí cho hệ hai điểm $(x,y)$ rõ ràng là một hệ thống
phụ thuộc vào $4$ tham số ( hệ thống $4$ chiều). Thật vậy các khả
năng vị trí điểm $x$ là một mặt cầu $S^{2}$ tâm $O$, ứng với mỗi
điểm $x$, khả năng vị trí điểm $y$ là một mặt cầu $S^{2}$ tâm $x$.
Như vậy hệ trên là một không gian $4$ chiều $S^{2}\times S^{2}$.
Để mô tả hệ thống cần $4$ tham số độc lập.

Việc nghiên cứu không gian nhiều chiều tổng quát hơn $\mathbb{R}^{n}$
chính là việc nghiên cứu về đa tạp. Bản thân $\mathbb{R}^{n}$ cũng
là một đa tạp. Ví dụ phía trên là một đa tạp $4$ chiều.

Trong môn học này ta quy ước $\mathbb{R}^{n}=\{x=(x_{1},x_{2},...,x_{n})\ |\ x_{1},x_{2},...,x_{n}\in\mathbb{R}\}$
với các cấu trúc không gian vector quen thuộc, metric Euclid quen
thuộc, và chuẩn Euclide: $\left\Vert x\right\Vert =\sqrt{x_{1}^{2}+x_{2}^{2}+...+x_{n}^{2}}$.
Chúng ta thường viết vectơ $x$ ở dạng cột vì thường xuyên nhân ma
trận. 


\chapter{Dạng vi phân trong không gian Euclid}


\section{Những tính chất cơ bản}

Các đại lượng $dx$, $dxdy$, $dx_{1}dx_{2}...dx_{n}$,... sẽ chưa
được định nghĩa mà để lại ở một chương sau. Trước tiên chúng ta hiểu
đại khái $dx$ như là một vô cùng bé của $x$ như ở bậc đại học.
\begin{example*}
Xét trong $\mathbb{R}^{2}$, $(x,y)\in\mathbb{R}^{2}$. Ta có tính
chất về dạng: 
\[
dxdy=-dydx,
\]
(tính phản đối xứng). Như vậy $dxdx=-dxdx$ suy ra $dxdx=0$. Ở đây
$0$ là cái gì? Nó có phải là số $0$ đơn thuần không? Ta tạm xem
nó như một kí hiệu hình thức.
\end{example*}
Trong $\mathbb{R}^{n}$ với $x=(x_{1},...,x_{n})$. Biểu thức 

\[
dx_{i_{1}}dx_{i_{2}}\dotsb dx_{i_{k}},\ 1\leq i_{1}\le i_{2}\le\dotsb\le i_{k}\leq n
\]
 được gọi là một \emph{dạng vi phân} bậc $k$. \index{dạng vi phân}

Dạng vi phân bậc $k$ thỏa mãn tính chất \emph{phản đối xứng} (anticommutative):
\[
\boxed{dxdy=-dydx.}
\]


Trong trường hợp tổng quát ta đổi chổ bất kì hai thành phần $dx_{i_{m}},dx_{i_{n}}$
công thức vẫn đúng: 
\[
dx_{i_{1}}\dotsb dx_{i_{m}}\dotsb dx_{i_{n}}\dotsb dx_{i_{k}}=-dx_{i_{1}}\dotsb dx_{i_{n}}\dotsb dx_{i_{m}}\dotsb dx_{i_{k}}.
\]
Chứng minh dành cho bạn đọc.
\begin{example*}
Trong $\mathbb{R}^{3}$ 
\[
dxdydz=-dxdzdy=dydzdx
\]
\[
dxdzdx=-dxdxdz=0.
\]

\end{example*}

\begin{rem}
Trong một dạng nếu có hai thành phần giống nhau thì dạng đó bằng không.

Trong $\mathbb{R}^{n}$ mọi $k$-dạng với $k>n$ đều bằng $0$.
\end{rem}

\subsection*{Phép nhân của dạng}

Trong $\mathbb{R}^{n}$ nếu $\omega_{1}$ và $\omega_{2}$ là hai
dạng bậc $k$ và $l$ thì ta có $\omega_{1}\omega_{2}$ lại là một
dạng bậc $k+l$.
\begin{example*}
Trong $\mathbb{R}^{3}$, $\omega_{1}=dx$ là một $1$-dạng; $\omega_{2}=dydz$
là một $2$-dạng; $\omega_{1}\omega_{2}=(dx)(dydz)=dxdydz$ là một
$3$-dạng.
\end{example*}
Phép nhân này có tính chất kết hợp: 
\[
(\omega_{1}\omega_{2})\omega_{3}=\omega_{1}(\omega_{2}\omega_{3})=\omega_{1}\omega_{2}\omega_{3}.
\]
 Phép nhân có phần tử không: 
\[
\omega\cdot0=0\cdot\omega=0.
\]

\begin{prop}
Nếu $\omega_{1}$ là một $k$-dạng và $\omega_{2}$ là một $l$-dạng
thì
\[
\boxed{\omega_{1}\omega_{2}=(-1)^{kl}\omega_{2}\omega_{1}.}
\]
\end{prop}
\begin{proof}
With 
\[
\omega_{1}=dx_{i_{1}}dx_{i_{2}}\dotsb dx_{i_{k}}
\]
 and 
\[
\omega_{2}=dx_{j_{1}}dx_{j_{2}}\dotsb dx_{j_{l}}
\]
 we get 
\begin{align*}
\omega_{1}\omega_{2} & =(dx_{i_{1}}dx_{i_{2}}\dotsb dx_{i_{k}})(dx_{j_{1}}dx_{j_{2}}\dotsb dx_{j_{l}})\\
 & =(-1)^{k}dx_{j_{1}}(dx_{i_{1}}dx_{i_{2}}\dotsb dx_{i_{k}})(dx_{j_{2}}\dotsb dx_{j_{l}})\\
 & =(-1)^{2k}dx_{j_{1}}dx_{j_{2}}(dx_{i_{1}}dx_{i_{2}}\dotsb dx_{i_{k}})(dx_{j_{3}}\dotsb dx_{j_{l}})\\
 & =(-1)^{kl}(dx_{j_{1}}dx_{j_{2}}\dotsb dx_{j_{l}})(dx_{i_{1}}dx_{i_{2}}\dotsb dx_{i_{k}})=(-1)^{kl}\omega_{2}\omega_{1}.
\end{align*}

\end{proof}
Tổng quát: một dạng bậc $k$ trên tập mở $U\subset\mathbb{R}^{n}$
là một biểu thức $fdx_{i_{1}}dx_{i_{2}}\dotsb dx_{i_{k}}$ trong đó
$f:U\rightarrow\mathbb{R}$.


\subsection*{Dạng vi phân trên $\mathbb{R}^{n}$.}

$U\in\mathbb{R}^{n}$, cho $x=(x_{1},...,x_{n})$. Dạng bậc không
($0$-dạng) là một hàm thực $f:U\subset\mathbb{R}^{n}\rightarrow\mathbb{R}$.
Dạng bậc $k$ ($k$-dạng) là biểu thức ${\displaystyle \sum fdx_{i_{1}}dx_{i_{2}}...dx_{i_{k}}}$
, $f:U\subset\mathbb{R}^{n}\rightarrow\mathbb{R}$.
\begin{example}
$dx_{1},dx_{2},...,dx_{n}$, $2dx_{1}+3dx_{2}$, $x_{1}^{2}x_{2}dx_{1}+x_{3}dx_{4}$
là dạng bậc $1$ trong $\mathbb{R}^{n}$. $x_{1}x_{2}dx_{i_{1}}dx_{i_{2}}...dx_{i_{n}}$
là dạng bậc $n$ trong $\mathbb{R}^{n}$. $xdxdy$, $(xy+1)dxdz$
là những dạng bậc $2$ trong $\mathbb{R}^{3}$.
\end{example}
Trên $\mathbb{R}^{2}$ một dạng bậc $1$ là một biểu thức: $Pdx+Qdy$
trong đó $P,Q:U\subset\mathbb{R}^{2}\rightarrow\mathbb{R}$. một dạng
bậc $2$ là một biểu thức: $Pdxdy+Qdydx=(P-Q)dxdy=Rdxdy$. Một dạng
bậc lớn hơn $2$ $=0$. Không gian $n$ chiều thì mọi dạng bậc lớn
hơn $n$ đều bằng $0$.
\begin{rem*}
Không thể cộng trừ hai dạng khác bậc. Chằng hạn như $Pdx+Qdy+Rdxdy$
là sai.
\end{rem*}
Nhân hai dạng: $\alpha={\displaystyle \sum fdx_{i_{1}}dx_{i_{2}}...dx_{i_{k}}}$
là một $k$-dạng. $\beta={\displaystyle \sum gdx_{j_{1}}dx_{j_{2}}...dx_{j_{l}}}$
là một $l$-dạng. Ta định nghĩa $\alpha\beta={\displaystyle \sum fgdx_{i_{1}}dx_{i_{2}}...dx_{i_{k}}dx_{j_{1}}dx_{j_{2}}...dx_{j_{l}}}$
là một $k+l$-dạng.
\begin{example}
\[
(dx)(dy)=dxdy
\]
 
\[
dx(dy+dz)=dxdy+dzdz
\]
\end{example}
\begin{prop}
Cho $\alpha$ và $\beta$ là hai dạng, ta có: 
\[
\alpha\beta=(-1)^{deg(\alpha).deg(\beta)}\beta\alpha
\]
 \end{prop}
\begin{proof}
Bạn đọc tự chứng minh.\end{proof}
\begin{cor}
Nhân một dạng bậc lẻ với chính nó thì bằng $0$.\end{cor}
\begin{proof}
Giả sử $\alpha$ là một $k$-dạng ($k$ lẻ). Ta có: $\alpha^{2}=\alpha.\alpha=(-1)^{k.k}\alpha.\alpha$
suy ra $\alpha^{2}=0$. 
\end{proof}

\section{Đạo hàm của dạng}
\begin{defn}
Cho $f:U\subset\mathbb{R}^{n}\rightarrow\mathbb{R}$, $U$ mở, $f$
trơn mọi cấp ( tức $f\in C^{\infty}(U)$). Ta gọi $f$ là dạng trơn
bậc $0$. Đạp hàm của dạng bậc $0$ được định nghĩa: 
\end{defn}
\begin{center}
$df=\dfrac{\partial f}{\partial x_{1}}dx_{1}+\dfrac{\partial f}{\partial x_{2}}dx_{2}+....+\dfrac{\partial f}{\partial x_{n}}dx_{n}={\displaystyle \sum_{i=1}^{n}\dfrac{\partial f}{\partial x_{i}}dx_{i}}$ 
\par\end{center}

Như vậy, đạo hàm của dạng bậc $0$ là dạng bậc $1$.
\begin{example}
Dạng bậc $0$ trên $\mathbb{R}$. 
\[
u=f(x)=x^{2}:\mathbb{R}\rightarrow\mathbb{R}
\]
 
\[
du=2xdx
\]
 $f$ hàm trên $\mathbb{R}$ 
\[
u=f(x)
\]
 
\[
du=f'(x)dx
\]

\end{example}

\begin{example}
Với 
\[
f:\mathbb{R}^{2}\rightarrow\mathbb{R}
\]
 
\[
(x,y)\mapsto f(x,y)
\]
 Ta có $df=\dfrac{\partial f}{\partial x}dx+\dfrac{\partial f}{\partial y}dy$.
\end{example}
Tổng quát:
\[
f:\mathbb{R}^{n}\rightarrow\mathbb{R}
\]
\[
(x_{1},..,x_{n})\mapsto f(x_{1},..,x_{n})
\]
 Ta có 
\[
df=\dfrac{\partial f}{\partial x_{1}}dx_{1}+\dfrac{\partial f}{\partial x_{2}}dx_{2}+....+\dfrac{\partial f}{\partial x_{n}}dx_{n}={\displaystyle \sum_{i=1}^{n}\dfrac{\partial f}{\partial x_{i}}dx_{i}}
\]
 

Phía trên đây là đạo hàm của $0$-dạng, bây giờ ta nói về đạo hàm
của một $k$-dạng.
\begin{defn}
Cho $\alpha={\displaystyle \sum fdx_{i_{1}}...dx_{i_{k}}}$ là một
$k$-dạng. Ta định nghĩa
\end{defn}
\[
d\alpha={\displaystyle \sum dfdx_{i_{1}}...dx_{i_{k}}}
\]
 là một $(k+1)$-dạng.

Đặt $\Omega^{k}(U)$ là tập hợp các dạng bậc $k$ trên $U$ thì $d:\Omega^{k}(U)\rightarrow\Omega^{k+1}(U)$.
\begin{example}
Trên $\mathbb{R}^{2}$ $\alpha=Pdx+Qdy$, ta có 
\[
d\alpha=d(Pdx+Qdy)=dPdx+dQdy=(\dfrac{\partial P}{\partial x}dx+\dfrac{\partial P}{\partial y}dy)dx+(\dfrac{\partial Q}{\partial x}dx+\dfrac{\partial Q}{\partial y}dy)dy=\dfrac{\partial P}{\partial y}dydx+\dfrac{\partial Q}{\partial x}dxdy=(\dfrac{\partial Q}{\partial x}-\dfrac{\partial P}{\partial y})dxdy
\]

\end{example}
Nhắc lại định lý Green trong giải tích cổ điển.

\begin{tikzpicture}[line cap=round,line join=round,>=triangle 45,x=1.0cm,y=1.0cm,scale=0.54]
\clip(2.92,-5.34) rectangle (17.08,6.32);
\draw [->] (4.,-3.) -- (4.,4.);
\draw [->] (2.,-1.) -- (10.,-1.);
\draw [rotate around={39.44440375479764:(6.616028203809059,1.3786909668423168)}] (6.616028203809059,1.3786909668423168) ellipse (1.780725937839661cm and 1.0147650118872027cm);
\draw (2.94,4.28) node[anchor=north west] {$\mathbb{R}^2$};
\draw [->] (6.923837578369623,0.35378598291576924) -- (7.180569788606215,0.5312680504204655);
\draw (6.94,2.48) node[anchor=north west] {$D$};
\draw (7.62,1.12) node[anchor=north west] {$\partial D$};
\end{tikzpicture}

\[
{\displaystyle \int_{D}\left(\dfrac{\partial Q}{\partial x}-\dfrac{\partial P}{\partial y}\right)dxdy={\displaystyle \int_{\partial D}Pdx+Qdy}}.
\]


Công thức Newton-Leibniz: 
\[
{\displaystyle \int_{[a,b]}du=u|_{\partial[a,b]}}
\]
.
\begin{example}
Trên $\mathbb{R}^{3}$ Cho dạng 
\[
\alpha=Pdydz+Qdzdx+Rdxdy
\]
 Thì 
\begin{align*}
d\alpha & =d(Pdydz+Qdzdx+Rdxdy)=dPdydz+dQdzdx+dRdxdy=\\
= & (\dfrac{\partial P}{\partial x}dx+\dfrac{\partial P}{\partial y}dy+\dfrac{\partial P}{\partial z}dz)dydz+(\dfrac{\partial Q}{\partial x}dx+\dfrac{\partial Q}{\partial y}dy+\dfrac{\partial Q}{\partial z}dz)dzdx+(\dfrac{\partial R}{\partial x}dx+\dfrac{\partial R}{\partial y}dy+\dfrac{\partial R}{\partial z}dz)dxdy\\
 & =(\dfrac{\partial P}{\partial x}+\dfrac{\partial Q}{\partial y}+\dfrac{\partial R}{\partial z})dxdydz.
\end{align*}
Cho dạng $\beta=Pdx+Qdy+Rdz$ thì 
\begin{align*}
d\beta & =d(Pdx+Qdy+Rdz)=dPdx+dQdy+dRdz=\\
= & (\dfrac{\partial P}{\partial x}dx+\dfrac{\partial P}{\partial y}dy+\dfrac{\partial P}{\partial z}dz)dx+(\dfrac{\partial Q}{\partial x}dx+\dfrac{\partial Q}{\partial y}dy+\dfrac{\partial Q}{\partial z}dz)dy+\\
+ & (\dfrac{\partial R}{\partial x}dx+\dfrac{\partial R}{\partial y}dy+\dfrac{\partial R}{\partial z}dz)dz=(\dfrac{\partial Q}{\partial x}-\dfrac{\partial P}{\partial y})dxdy+(\dfrac{\partial R}{\partial y}-\dfrac{\partial Q}{\partial z})dydz+(\dfrac{\partial P}{\partial z}-\dfrac{\partial R}{\partial x})dzdx.
\end{align*}

\end{example}
Nhắc lại trong giải tích cổ điển: Định lý Stokes trong $\mathbb{R}^{3}$
\[
{\displaystyle \int_{\partial S}Pdx+Qdy+Rdz={\displaystyle \int_{S}(\dfrac{\partial Q}{\partial x}-\dfrac{\partial P}{\partial y})dxdy+(\dfrac{\partial R}{\partial y}-\dfrac{\partial Q}{\partial z})dydz+(\dfrac{\partial P}{\partial z}-\dfrac{\partial R}{\partial x})dzdx}}.
\]
 Định lý Gauss - Ostragradski: 
\[
{\displaystyle \int_{\partial E}Pdydz+Qdzdx+Rdxdy={\displaystyle \int_{E}(\dfrac{\partial P}{\partial x}+\dfrac{\partial Q}{\partial y}+\dfrac{\partial R}{\partial z})dxdydz}}.
\]


Ta để ý các công thức Newton-Leibniz, Green, Stokes, Gauss - Ostragradski
đều có dạng 
\[
{\displaystyle \int_{\partial C}\alpha={\displaystyle \int_{C}d\alpha}}.
\]
Trong đó $\alpha$ là một dạng và $d\alpha$ là đạo hàm của dạng đó.
Ở phần sau ta sẽ tổng quát hóa lên thành định lý Stokes trong không
gian nhiều chiều và trên đa tạp.

Trong giải tích cổ điển:

Cho $F=(P,Q,R)$, ta có khái niệm: 
\[
DivF=\dfrac{\partial P}{\partial x}+\dfrac{\partial Q}{\partial y}+\dfrac{\partial R}{\partial z}.
\]
\[
RotF=CurlF=(\dfrac{\partial Q}{\partial x}-\dfrac{\partial P}{\partial y},\dfrac{\partial R}{\partial y}-\dfrac{\partial Q}{\partial z},\dfrac{\partial P}{\partial z}-\dfrac{\partial R}{\partial x}).
\]

\begin{thm*}
Đạo hàm của đạo hàm của một dạng bất kì bằng $0$. (Ở đây hàm lấy
giá trị thực $f$ phải trơn.) Tức $\alpha={\displaystyle \sum fdx_{i_{1}}dx_{i_{2}}...dx_{i_{k}}}$
là một $k$-dạng. Ta có: $d(d\alpha)=0$. \end{thm*}
\begin{proof}
Phần chứng minh giành cho bạn đọc.\end{proof}
\begin{example}
Trong $\mathbb{R}^{3}$ cho $\alpha=Pdx+Qdy+Rdz$

Như trên $d\alpha=(\dfrac{\partial Q}{\partial x}-\dfrac{\partial P}{\partial y})dxdy+(\dfrac{\partial R}{\partial y}-\dfrac{\partial Q}{\partial z})dydz+(\dfrac{\partial P}{\partial z}-\dfrac{\partial R}{\partial x})dzdx$.

Do đó 
\begin{align*}
d(d\alpha) & =d\left((\dfrac{\partial Q}{\partial x}-\dfrac{\partial P}{\partial y})dxdy+(\dfrac{\partial R}{\partial y}-\dfrac{\partial Q}{\partial z})dydz+(\dfrac{\partial P}{\partial z}-\dfrac{\partial R}{\partial x})dzdx\right)=\\
= & d\left(\dfrac{\partial Q}{\partial x}-\dfrac{\partial P}{\partial y}\right)dxdy+d\left(\dfrac{\partial R}{\partial y}-\dfrac{\partial Q}{\partial z}\right)dydz+d\left(\dfrac{\partial P}{\partial z}-\dfrac{\partial R}{\partial x}\right)dzdx\\
= & \left(\dfrac{\partial^{2}Q}{\partial x^{2}}dx+\dfrac{\partial^{2}Q}{\partial x\partial y}dy+\dfrac{\partial^{2}Q}{\partial x\partial z}dz-\dfrac{\partial^{2}P}{\partial y\partial x}dx-\dfrac{\partial^{2}P}{\partial y^{2}}dy-\dfrac{\partial^{2}P}{\partial y\partial z}dz\right)dxdy+dydz\\
 & +\left(\dfrac{\partial^{2}R}{\partial y\partial x}dx+\dfrac{\partial^{2}R}{\partial y^{2}}dy+\dfrac{\partial^{2}R}{\partial y\partial z}dz-\dfrac{\partial^{2}Q}{\partial z\partial x}dx-\dfrac{\partial^{2}Q}{\partial z\partial y}dy-\dfrac{\partial^{2}Q}{\partial z^{2}}dz\right)\\
 & +\left(\dfrac{\partial^{2}P}{\partial z\partial x}dx+\dfrac{\partial^{2}P}{\partial z\partial y}dy+\dfrac{\partial^{2}P}{\partial z^{2}}dz-\dfrac{\partial^{2}R}{\partial x^{2}}dx-\dfrac{\partial^{2}R}{\partial x\partial y}dy-\dfrac{\partial^{2}R}{\partial x\partial z}dz\right)dzdx.\\
= & \dfrac{\partial^{2}Q}{\partial x\partial z}dzdxdy-\dfrac{\partial^{2}P}{\partial y\partial z}dzdxdy+\dfrac{\partial^{2}R}{\partial y\partial x}dxdydz-\dfrac{\partial^{2}Q}{\partial z\partial x}dxdydz+\dfrac{\partial^{2}P}{\partial z\partial y}dydzdx-\dfrac{\partial^{2}R}{\partial x\partial y}dydzdx\\
= & 0.
\end{align*}

\end{example}

\section{Dạng khớp và dạng đóng}
\begin{defn}
$\alpha$ được gọi là dạng khớp nếu có dạng $\beta$ sao cho $d\beta=\alpha$. 
\end{defn}
Một cách nôm na dạng khớp là đạo hàm của dạng khác, tức dạng có nguyên
hàm.

Cho $\alpha$ là dạng bậc một trên $\mathbb{R}$ ($\alpha\in\Omega^{1}(\mathbb{R}$)
thì $\alpha=fdx$ ; $f\in C^{\infty}(\mathbb{R})$.

$\alpha$ khớp $\Leftrightarrow$ có tồn tại hàm $F\in\Omega^{0}(\mathbb{R})=C^{\infty}(\mathbb{R})$
sao cho $dF=fdx$ tức $F'dx=fdx$ tức $F'=f$. Nghĩa là $F$ là một
nguyên hàm của $f$.

Như vậy dạng khớp như là tổng quát hóa của khái niệm có nguyên hàm
trong giải tích cổ điển.

Trong $\mathbb{R}^{2}$ cho $\alpha=Pdx+Qdy\in\Omega^{1}(\mathbb{R}^{2})$

$\alpha$ là khớp $\Leftrightarrow$ tồn tại $F\in\Omega^{0}(\mathbb{R}^{2})=C^{\infty}(\mathbb{R}^{2})$
sao cho 
\[
dF=\alpha\Leftrightarrow\dfrac{\partial F}{\partial x}dx+\dfrac{\partial F}{\partial y}dy=Pdx+Qdy\Leftrightarrow\left\{ \begin{array}{lll}
\dfrac{\partial F}{\partial x}=P\\
\dfrac{\partial F}{\partial y}=Q
\end{array}\right.\Leftrightarrow(P,Q)=(\dfrac{\partial F}{\partial x},\dfrac{\partial F}{\partial y})=\nabla F=gradF
\]
 Trong giải tích hàm nhiều biến, trường $(P,Q)$ sao cho $\exists F:\nabla F=(P,Q)$
thì $(P,Q)$ gọi là bảo toàn. Nếu $\alpha$ là khớp thì $(P,Q)$ là
bảo toàn và hàm $F$ được gọi là hàm thế. Bài toán quan tâm là một
dạng có khớp hay không, nếu khớp tìm nguyên hàm?
\begin{example}
Trong $\mathbb{R}^{2}$ cho $\alpha=xydx+\dfrac{1}{2}x^{2}dy$ là
$1$-dạng. Hỏi $\alpha$ có khớp hay không? (Tức trường $(xy,\dfrac{1}{2}x^{2})$
có bảo toàn hay không.)

Đây thực ra là bài toán tìm nguyên hàm. Tìm $F$ sao cho 
\[
\left\{ \begin{array}{lll}
\dfrac{\partial F}{\partial x}=xy &  & (1)\\
\dfrac{\partial F}{\partial y}=\dfrac{1}{2}x^{2} &  & (2)
\end{array}\right.
\]


$(1)\Rightarrow F(x,y)={\displaystyle \int xydx=\dfrac{1}{2}x^{2}y+C(y)}$.
Thay vào $(2)$ ta được:
\[
(\dfrac{1}{2}x^{2}y+C(y))'_{y}=\dfrac{1}{2}x^{2}\Leftrightarrow\dfrac{1}{2}x^{2}+C'(y)=\dfrac{1}{2}x^{2}
\]
 tức $C'(y)=0\Rightarrow C(y)=C$. Vậy $F(x,y)=\dfrac{1}{2}x^{2}+C$.
Vậy trường $(xy,\dfrac{1}{2}x^{2})$ là trường bảo toàn hay $\alpha$
là dạng khớp và hàm thế $F=\dfrac{1}{2}x^{2}+C$ ($C$ là hằng số
không phụ thuộc cả $x$ lẩn $y$). Ví dụ trên là tương đối đơn giản,
trong trường hợp tổng quát ta cần phương pháp lý thuyết.
\end{example}
\textit{Điều kiện cần để khớp}: Giả sử $\alpha$ khớp thì $\exists\beta:d\beta=\alpha\Rightarrow d(\alpha)=d(d\beta)=0$.
Như vậy một dạng khớp thì có đạo hàm bằng $0$. Nhắc lại trong giải
tích hàm nhiều biến $\alpha=Pdx+Qdy$, $d\alpha=(\dfrac{\partial Q}{\partial x}-\dfrac{\partial P}{\partial y})dxdy$.
Như vậy $(P,Q)$ bào toàn thì $\dfrac{\partial Q}{\partial x}=\dfrac{\partial P}{\partial y}$.
\begin{defn}
Nếu $d\alpha=0$ ta nói $\alpha$ là dạng đóng. Như vậy một dạng khớp
thì đóng và không có chiều ngược lại. Trong giải tích hàm nhiều biến
ta quan tâm bài toán trường $(P,Q)$ có $\dfrac{\partial Q}{\partial x}=\dfrac{\partial P}{\partial y}$
mà $(P,Q)$ không bảo toàn.
\end{defn}
Như vậy mối quan hệ giữa khớp và đóng là phong phú. 


\subsection*{Bài tập}
\begin{problem}
Hệ thống Lotka-Volterra là một mô hình con thú săn mồi-con mồi. Nó
là một cặp phương trình vi phân 
\[
\dfrac{dx}{dt}=-rx+sxy,\dfrac{dy}{dt}=py-qxy
\]
Ở đây $x(t)$ đại diện cho dân số con thú săn mồi, $y(t)$ đại diện
cho dân số con mồi tại thời điểm $t$, trong đó $p,q,s,t$ là các
hằng số dương. Vấn đề chúng ta quan tâm là việc giải đường cong nghiệm
(còn gọi là quỹ đạo) $(x(t),y(t))$ của hệ thống này. (\cite{Sjamaar06})
\end{problem}

\chapter{Kéo lui của một dạng}


\section{Công thức đổi biến trong $\mathbb{R}$}

Nhắc lại trong không gian một chiều $\mathbb{R}$ ta có công thức
đổi biến tích phân như sau: ${\displaystyle \int f(u)u'(x)dx={\displaystyle \int f(u)du}}$\\
 Chẳng hạn ${\displaystyle \int\sin^{2}t\cos tdt}$\\
Đổi biến $u=\sin t\Rightarrow du=\cos tdt$ do đó, ${\displaystyle \int\sin^{2}t\cos tdt={\displaystyle \int u^{2}du=\dfrac{u^{3}}{3}+C}}$. 


\section{Kéo lui của một dạng}

Mục đích chính của chương này là tổng quát hóa công thức trên trong
trường hợp $n$-chiều. 

\definecolor{qqqqff}{rgb}{0.,0.,1.} \begin{tikzpicture}[line cap=round,line join=round,>=triangle 45,x=1.0cm,y=1.0cm]
\clip(2.1,-0.4) rectangle (8.44,1.98);
\draw [rotate around={16.232350661156207:(3.23,0.39)}] (3.23,0.39) ellipse (1.0255857283809675cm and 0.6122304192530148cm);
\draw [rotate around={17.818888914522773:(7.14,0.49)}] (7.14,0.49) ellipse (1.0972296049664032cm and 0.6522367714371142cm);
\draw (3.88,1.68) node[anchor=north west] {$\mathbb{R}^n$};
\draw (7.58,1.9) node[anchor=north west] {$\mathbb{R}^n$};
\draw (2.92,0.72) node[anchor=north west] {$x$};
\draw (7.26,1.04) node[anchor=north west] {$u$};
\draw (3.34,0.22)-- (7.26,0.44);
\draw [->] (5.020319692762799,0.3143036562264836) -- (5.360371071956821,0.3333881724057399);
\draw (5.22,0.88) node[anchor=north west] {$\varphi$};
\begin{scriptsize}
\draw [fill=qqqqff] (3.34,0.22) circle (1.5pt);
\draw [fill=qqqqff] (7.26,0.44) circle (1.5pt);
\end{scriptsize}
\end{tikzpicture} 

$\omega$ là dạng theo biến $x$ trong $\mathbb{R}^{n}$, câu hỏi
đặt ra: mối quan hệ vi phân của $\omega$ theo $x$ và theo $u$ như
thế nào? Giả sử $\omega$ là dạng theo $x$, đổi biến $u=\varphi(x)$
thì $\omega$ trở thành dạng gì?

\definecolor{qqqqff}{rgb}{0.,0.,1.} \begin{tikzpicture}[line cap=round,line join=round,>=triangle 45,x=1.0cm,y=1.0cm]
\clip(2.06,-0.56) rectangle (8.4,1.72);
\draw [rotate around={16.232350661156207:(3.23,0.39)}] (3.23,0.39) ellipse (1.0255857283809675cm and 0.6122304192530148cm);
\draw [rotate around={17.818888914522773:(7.14,0.49)}] (7.14,0.49) ellipse (1.0972296049664032cm and 0.6522367714371142cm);
\draw (3.88,1.68) node[anchor=north west] {$U$};
\draw (7.58,1.9) node[anchor=north west] {$V$};
\draw (2.92,0.72) node[anchor=north west] {$x$};
\draw (7.26,1.04) node[anchor=north west] {$y$};
\draw (3.34,0.22)-- (7.26,0.44);
\draw [->] (5.020319692762799,0.3143036562264836) -- (5.360371071956821,0.3333881724057399);
\draw (5.22,0.88) node[anchor=north west] {$\varphi$};
\draw (7.1,0.52) node[anchor=north west] {$\omega$};
\begin{scriptsize}
\draw [fill=qqqqff] (3.34,0.22) circle (1.5pt);
\draw [fill=qqqqff] (7.26,0.44) circle (1.5pt);
\end{scriptsize}
\end{tikzpicture} 

$\omega$ là dạng trên tập mở $V$. Câu hỏi đặt ra trên $U$ có dạng
gì tương ứng? Dạng tương ứng đó gọi là dạng kéo lui của dạng $\omega$.
\begin{defn}
Kéo lui (pull-back) của dạng $\omega$ (kí hiệu: $\varphi^{*}\omega$)
là một dạng trên $U$ xác định như sau: Nếu 
\[
\omega={\displaystyle \sum fdy_{i_{1}}...dy_{i_{k}}}
\]
\[
\varphi:\mathbb{R}^{n}\longrightarrow\mathbb{R}^{n}
\]
\[
x(x_{1},...,x_{n})\longmapsto\varphi(\varphi_{1},...,\varphi_{n})=y(y_{1},...,y_{n})
\]
 Thì $\varphi^{*}\omega={\displaystyle \sum f\circ\varphi d\varphi_{i_{1}}...d\varphi_{i_{k}}}$
\end{defn}
Trong đó:
\begin{itemize}
\item $\omega$ là dạng bậc $k$ theo biến $y$
\item $f$ là hàm theo biến $y$
\item $\varphi$ là ánh xạ đi từ $\mathbb{R}^{n}$ vào $\mathbb{R}^{n}$
biến $x$ thành $y$
\item $\varphi:\mathbb{R}^{n}\longrightarrow\mathbb{R}^{n}$
\item $x\longmapsto y$
\item Chúng ta có thể hiểu $\varphi^{*}\omega$ nhận được bằng cách thay
$y$ bởi $\varphi(x)$ và thay $u_{i}$ bởi $\varphi_{i}(x)$. \textbf{ }\end{itemize}
\begin{example}
\[
\varphi:\mathbb{R}\rightarrow\mathbb{R}
\]
\[
x\mapsto u=\varphi
\]
 $\omega=f(u)du$ thì $\varphi^{*}\omega=f(\varphi(x))d\varphi=f(\varphi(x))\varphi'(x)dx$. 
\end{example}
Xem trường hợp nhiều chiều hơn: $\varphi:\mathbb{R}^{2}\rightarrow\mathbb{R}^{2}(x,y)\longmapsto(u,v)$
ở đây $u,v$ là hàm theo biến $x,y$. Cho $\omega=du$ thì $\varphi^{*}\omega=du=\dfrac{\partial u}{\partial x}dx+\dfrac{\partial u}{\partial y}dy$
Cho $\beta=dudv$ thì 
\begin{align*}
\varphi^{*}\beta & =dudv=d\varphi_{1}d\varphi_{2}=(\dfrac{\partial u}{\partial x}dx+\dfrac{\partial u}{\partial y}dy)(\dfrac{\partial v}{\partial x}dx+\dfrac{\partial v}{\partial y}dy)=(\dfrac{\partial u}{\partial x}\dfrac{\partial v}{\partial y}-\dfrac{\partial u}{\partial y}\dfrac{\partial v}{\partial x})dxdy\\
= & det\left(\begin{array}{ccc}
\dfrac{\partial u}{\partial x} & \dfrac{\partial u}{\partial y}\\
\dfrac{\partial v}{\partial x} & \dfrac{\partial v}{\partial y}
\end{array}\right)dxdy=(detJ\varphi)dxdy=\dfrac{\varphi(u,v)}{\varphi(x,y)}dxdy.
\end{align*}
 (dấu bằng cuối cùng là cách kí hiệu trong giải tích cổ điển).

\definecolor{qqqqff}{rgb}{0.,0.,1.} \begin{tikzpicture}[line cap=round,line join=round,>=triangle 45,x=1.0cm,y=1.0cm]
\clip(1.24,-1.36) rectangle (8.4,1.88);
\draw [rotate around={16.232350661156207:(3.23,0.39)}] (3.23,0.39) ellipse (1.0255857283809675cm and 0.6122304192530148cm);
\draw [rotate around={17.818888914522773:(7.14,0.49)}] (7.14,0.49) ellipse (1.0972296049664032cm and 0.6522367714371142cm);
\draw (3.88,1.68) node[anchor=north west] {$\mathbb{R}^2$};
\draw (7.58,1.9) node[anchor=north west] {$\mathbb{R}^2$};
\draw (2.76,0.88) node[anchor=north west] {$(x,y)$};
\draw (7.2,1.2) node[anchor=north west] {$(u,v)$};
\draw (3.38,0.06)-- (7.26,0.44);
\draw [->] (5.043173573448893,0.2228881334821081) -- (5.3797550406103225,0.2558522977917327);
\draw (5.22,0.88) node[anchor=north west] {$\varphi$};
\draw (1.62,-0.1) node[anchor=north west] {$\varphi^*\omega=\dfrac{\varphi (u,v)}{\varphi (x,y)}dxdy$};
\draw (6.46,-0.24) node[anchor=north west] {$\omega = dudv$};
\begin{scriptsize}
\draw [fill=qqqqff] (3.38,0.06) circle (1.5pt);
\draw [fill=qqqqff] (7.26,0.44) circle (1.5pt);
\end{scriptsize}
\end{tikzpicture} 

Nhớ lại công thức đổi biến trong tích phân hàm hai biến số: 
\[
{\displaystyle \int\int f(u,v)dudv={\displaystyle \int\int f(u(x,y),v(x,y))\dfrac{\varphi(u,v)}{\varphi(x,y)}dxdy}}.
\]
 Ở đây cận (miền) lấy tích phân ở hai vế thuận theo qui tắc đổi cận
tích phân khi ta đổi biến. Biểu thức dưới dấu tích phân trong vế phải
$(*)$ là dạng kéo lui của biểu thức dưới dấu tích phân trong vế trái
$(*)$.
\begin{example}
$n$-chiều

\definecolor{qqqqff}{rgb}{0.,0.,1.} \begin{tikzpicture}[line cap=round,line join=round,>=triangle 45,x=1.0cm,y=1.0cm]
\clip(2.1,-0.4) rectangle (8.44,1.98);
\draw [rotate around={16.232350661156207:(3.23,0.39)}] (3.23,0.39) ellipse (1.0255857283809675cm and 0.6122304192530148cm);
\draw [rotate around={17.818888914522773:(7.14,0.49)}] (7.14,0.49) ellipse (1.0972296049664032cm and 0.6522367714371142cm);
\draw (3.88,1.68) node[anchor=north west] {$\mathbb{R}^n$};
\draw (7.58,1.9) node[anchor=north west] {$\mathbb{R}^n$};
\draw (2.92,0.72) node[anchor=north west] {$x$};
\draw (7.26,1.04) node[anchor=north west] {$u$};
\draw (3.34,0.22)-- (7.26,0.44);
\draw [->] (5.020319692762799,0.3143036562264836) -- (5.360371071956821,0.3333881724057399);
\draw (5.22,0.88) node[anchor=north west] {$\varphi$};
\begin{scriptsize}
\draw [fill=qqqqff] (3.34,0.22) circle (1.5pt);
\draw [fill=qqqqff] (7.26,0.44) circle (1.5pt);
\end{scriptsize}
\end{tikzpicture} 

$\omega=du_{1}du_{2}...du_{n}$ 
\begin{align*}
\varphi^{*}\omega & =d\varphi_{1}d\varphi_{2}...d\varphi_{n}=du_{1}du_{2}...du_{n}=\left({\displaystyle \sum_{i=1}^{n}\dfrac{\partial u_{1}}{\partial u_{i}}dx_{i}}\right)\left({\displaystyle \sum_{i=1}^{n}\dfrac{\partial u_{2}}{\partial u_{i}}dx_{i}}\right)...\left({\displaystyle \sum_{i=1}^{n}\dfrac{\partial u_{n}}{\partial u_{i}}dx_{i}}\right)\\
= & det\left(\begin{array}{ccc}
\dfrac{\partial\varphi_{1}}{\partial x_{1}} & ... & \dfrac{\partial\varphi_{1}}{\partial x_{n}}\\
\dfrac{\partial\varphi_{2}}{\partial x_{1}} & ... & \dfrac{\partial\varphi_{2}}{\partial x_{n}}\\
. & ... & .\\
. & ... & .\\
. & ... & .\\
\dfrac{\partial\varphi_{n}}{\partial x_{1}} & ... & \dfrac{\partial\varphi_{n}}{\partial x_{n}}
\end{array}\right)dx_{1}dx_{2}...dx_{n}=det(J\varphi)dx_{1}dx_{2}...dx_{n}.\\
\end{align*}
 Tóm lại: $\varphi:\mathbb{R}^{n}\rightarrow\mathbb{R}^{n}$, $x\mapsto u$,
\[
\varphi^{*}(du_{1}du_{2}...du_{n})=det(J\varphi)dx_{1}dx_{2}...dx_{n}.
\]
 Ta có công thức đổi biến tích phân: 
\end{example}
\begin{center}
${\displaystyle \int\int...\int f(u)du_{1}du_{2}...du_{n}={\displaystyle \int\int...\int f(\varphi(x))det(J\varphi)dx_{1}dx_{2}...dx_{n}}}.$ 
\par\end{center}

Trong đó cận lấy tích phân tương ứng ở hai vế tuân theo qui luật đổi
biến.
\begin{thm}
Ta có một số tính chất sau đây của dạng kéo lui 
\begin{enumerate}
\item $\varphi^{*}(a\omega_{1}+b\omega_{2})=a\varphi^{*}\omega_{1}+b\varphi^{*}\omega_{2}$.
(Tính tuyến tính) 
\item $\varphi^{*}(\omega_{1}.\omega_{2})=\varphi^{*}\omega_{1}.\varphi^{*}\omega_{2}$.
(Kéo lui của tích bằng tích kéo lui) 
\item $(\psi\circ\varphi)^{*}\omega=\varphi^{*}(\psi^{*}\omega)$. (Phép
kéo lui thực hiện từng bước một)\\
 \begin{tikzpicture}[line cap=round,line join=round,>=triangle 45,x=1.0cm,y=1.0cm]
		\clip(-0.58,0.88) rectangle (10.3,2.7);
		\draw [rotate around={168.87964676637708:(1.0827950287934676,1.7247826824523533)}] (1.0827950287934676,1.7247826824523533) ellipse (1.2448106666071814cm and 0.4089141478761202cm);
		\draw [rotate around={169.83333115421928:(5.21115768157666,1.6659689578175532)}] (5.21115768157666,1.6659689578175532) ellipse (1.1403299638718054cm and 0.4615241904518754cm);
		\draw [rotate around={164.6123021629423:(8.964354985554843,1.720615043191245)}] (8.964354985554843,1.720615043191245) ellipse (1.1636884188726422cm and 0.5264082930489923cm);
		\draw (2.22,1.68)-- (4.141097415533654,1.6807700328950097);
		\draw (6.257882976398445,1.681592628237431)-- (7.993175754381506,1.6525428478712936);
		\draw [->] (2.9000079073408465,1.680272567363469) -- (3.2599998329090027,1.6804168628179232);
		\draw [->] (6.859523242633884,1.671520830662224) -- (7.20043213797054,1.6658138233721325);
		\draw (3.04,2.38) node[anchor=north west] {$\varphi$};
		\draw (7.02,2.42) node[anchor=north west] {$\psi$};
		\end{tikzpicture} 
\item $\varphi^{*}(d\omega)=d(\varphi^{*}\omega)$. (Đạo hàm của dạng kéo
lui bằng dạng kéo lui của đạo hàm).
\end{enumerate}
\end{thm}
Ta làm rõ tính chất 4: Ta có $\varphi^{*}\omega$ và $\omega$ là
dạng bậc $k$, còn $\varphi^{*}d\omega$ và $d\omega$ là dạng bậc
$k+1$.

Kí hiệu: $\Omega^{k}(V)$ là tập hợp tất cả các dạng (trơn) bậc $k$
trên $V$.

\[
\varphi:U\rightarrow V
\]
\[
\varphi^{*}:\Omega^{k}(V)\rightarrow\Omega^{k}(U)
\]
\[
d:\Omega^{k}(V)\rightarrow\Omega^{k+1}(V)
\]
\[
\Omega^{k}(U)\rightarrow\Omega^{k+1}(U)
\]
 Ta có sơ đồ giao hoán: 
\[
\xymatrix{\Omega^{k}(V)\ar[r]^{d}\ar[d]_{\varphi^{*}} & \Omega^{k+1}(V)\ar[d]^{\varphi^{*}}\\
\Omega^{k}(U)\ar[r]_{d} & \Omega^{k+1}(U)
}
\]
\[
d\circ\varphi^{*}=\varphi^{*}\circ d
\]
 Phần chứng minh giành cho bạn đọc. 


\chapter{Tích phân của dạng}

Trong chương này ta xây dựng khái niệm tích phân của dạng trên một
tập con (một miền được tham số hóa - parametrized region) của $\mathbb{R}^{n}$,
là sự tổng quát hóa của các khái niệm tích phân đường và tích phân
mặt quen thuộc. Từ đó đưa ra phát biểu tổng quát hơn của định lý Stokes.

Dưới đây để đơn giản, ta sẽ xét miền lấy tích phân là những miền đơn
giản như hình hộp $R\subset\mathbb{R}^{k}$, có dạng $R=\left[a_{1},b_{1}\right]\times\left[a_{2},b_{2}\right]\cdots\left[a_{k},b_{k}\right]$,
rồi đến hình hộp suy biến (singular block, sẽ được định nghĩa sau).
Nhắc lại rằng với hàm $f:R\rightarrow\mathbb{R}^{k}$ trơn trên $R$\footnote{Với $V$ là một tập không mở, nếu không nói gì thêm ta hiểu rằng nói
$f$ trơn trên $V$ nghĩa là $f$ trơn trên một tập mở chứa $V$.} thì tích phân của $f$ trên $R$ tồn tại (theo cả nghĩa Riemann lẫn
Lebesgue). Từ định lý Fubini, ta có:

\begin{center}
${\displaystyle \ensuremath{\int_{R}{f}}=\ensuremath{\int_{{{a}_{k}}}^{{{b}_{k}}}\left({\cdots\left(\int_{{{a}_{2}}}^{{{b}_{2}}}{\left(\int_{{{a}_{1}}}^{{{b}_{1}}}{f\left({{x}_{1}},{{x}_{2}},\ldots,{{x}_{k}}\right)d{{x}_{1}}}\right)}d{{x}_{2}}\right)\cdots}\right)d{{x}_{k}}}}$.
\par\end{center}


\section{Tích phân của dạng}
\begin{defn}
(Tích phân dạng bậc $k$ trên hình hộp $R\subset\mathbb{R}^{k}$).
Cho $\omega$ là một dạng bậc $k$ xác định trên tập mở $U\subset\mathbb{R}^{k}$
và $U\supset R$. Do $\omega$ là dạng bậc $k$ xác định trên một
tập con của $R^{k}$ nên $\omega$ viết được thành $\omega=fdx_{1}dx_{2}\cdots dx_{k}$.
Ta định nghĩa 
\[
{\displaystyle \ensuremath{\int_{R}{\omega}}}\coloneqq{\displaystyle \ensuremath{\int_{R}{f}}}.
\]

\end{defn}
Từ định nghĩa trên ta có

\begin{center}
${\displaystyle \ensuremath{\int_{R}{-\omega}}}={\displaystyle \ensuremath{\int_{R}{-f}}}=-{\displaystyle \ensuremath{\int_{R}{f}}}=-{\displaystyle \ensuremath{\int_{R}{\omega}}}$.
\par\end{center}
\begin{rem}
Chú ý rằng trong định nghĩa trên ta phải quy $\omega$ về dạng sao
cho tích các $dx_{i}$ luôn là $dx_{1}dx_{2}\cdots dx_{k}$. Xét chẳng
hạn tích phân của dạng $\omega=f\left(x,y\right)dxdy$ trên $R=\left[a_{1},b_{1}\right]\times\left[a_{2},b_{2}\right]\subset\mathbb{R}^{2}$,
với $f$ khả tích trên $R$. Do $f\left(x,y\right)dxdy=-f\left(x,y\right)dydx$,
nên theo dịnh nghĩa trên, ta có:

\begin{equation}
{\displaystyle \ensuremath{\int_{R}{f\left(x,y\right)dxdy}}}={\displaystyle \ensuremath{\int_{R}{-f\left(x,y\right)dydx}}}=-{\displaystyle \ensuremath{\int_{R}{f\left(x,y\right)dydx}}}.
\end{equation}
Như vậy hai dạng bậc $k$ về hình thức có thể có cùng hàm hệ số $f$
nhưng nếu thứ tự các $dx_{i}$ khác nhau thì tích phân của chúng trên
$R\subset\mathbb{R}^{k}$ chưa chắc bằng nhau.

Bên cạnh đó ta biết rằng việc thay đổi thứ tự lấy tích phân của $f$
trên $R$ không ảnh hưởng gì, tức là:

\begin{equation}
{\displaystyle \ensuremath{\int_{a_{2}}^{b_{2}}\int_{a_{1}}^{b_{1}}{f\left(x,y\right)dxdy}}}={\displaystyle \ensuremath{\int_{a_{1}}^{b_{1}}\int_{a_{2}}^{b_{2}}{f\left(x,y\right)dydx}}}.
\end{equation}
Điều này không có gì mâu thuẫn với $\left(3.1.1\right)$, bạn đọc
kiểm tra xem như bài tập (xem ví dụ 3.8).\end{rem}
\begin{defn}
Cho $U$ là tập mở trong $\mathbb{R}^{n}$ và $\omega$ là một dạng
bậc $k$ xác định trên $U$. Cho hình hộp $R=\left[a_{1},b_{1}\right]\times\left[a_{2},b_{2}\right]\cdots\left[a_{k},b_{k}\right]$
trong $\mathbb{R}^{k}$ và $\varphi:R\rightarrow\mathbb{R}^{n}$ là
ánh xạ trơn xác định trên $R\subset\mathbb{R}^{k}$ sao cho $\varphi\left(R\right)\subset U$.
Khi đó nếu $n\geq k$ thì dạng kéo lui $\varphi^{*}\omega$ là một
dạng bậc $k$ xác định trên $R$, nếu $n<k$ thì $\omega$ (do đó
cả $\varphi^{*}\omega$) chỉ có thể bằng $0$. Ta định nghĩa tích
phân của $\omega$ trên $\varphi$, ký hiệu $\intop_{\varphi}\omega$,
là
\[
{\displaystyle \ensuremath{\int_{\varphi}{\omega}}}\coloneqq{\displaystyle \ensuremath{\int_{R}{\varphi^{*}\omega}}}.
\]

\end{defn}
Ánh xạ $\varphi$ trên đươc gọi là một hình hộp suy biến $k$-chiều.
Ảnh $\varphi\left(R\right)$ của nó có thể rất khác so với $R$ (do
$\varphi$ chỉ yêu cầu trơn), chẳng hạn $\varphi$ có thể mang đoạn
$\left[a,b\right]$ thành một đoạn cong, thậm chí trở thành một điểm. 
\begin{example}
(Tích phân đường). Cho $c:\left[a,b\right]\rightarrow\mathbb{R}^{3}$,
$c\left(t\right)=\left(x\left(t\right),y\left(t\right),z\left(t\right)\right)$
là một đường trơn và $\alpha=P\left(x,y,z\right)dx+Q\left(x,y,z\right)dy+R\left(x,y,z\right)dz$
là dạng bậc $1$ xác định trên một tập mở $U$ chứa $c\left(\left[a,b\right]\right)$.
Theo định nghĩa trên, ta có:
\begin{alignat*}{1}
{\displaystyle \ensuremath{\int_{c}{\alpha}}} & \coloneqq{\displaystyle \ensuremath{\int_{\left[a,b\right]}{c^{*}\alpha}}}\\
 & =\int_{\left[a,b\right]}\left(\left(P\circ c\right)dx\left(t\right)+\left(Q\circ c\right)dy\left(t\right)+\left(R\circ c\right)dz\left(t\right)\right)\\
 & =\int_{a}^{b}\left[\left(\left(P\circ c\right)x'\left(t\right)+\left(Q\circ c\right)y'\left(t\right)+\left(R\circ c\right)z'\left(t\right)\right)\right]dt.
\end{alignat*}

\end{example}

\begin{example}
(Tích phân mặt). Cho $D\subset\mathbb{R}^{2}$, $\varphi:D\rightarrow\mathbb{R}^{3}$,
$\varphi\left(u,v\right)=\left(x\left(u,v\right),y\left(u,v\right),z\left(u,v\right)\right)$
là một mặt trơn và $\alpha=P\left(x,y,z\right)dydz+Q\left(x,y,z\right)dzdx+R\left(x,y,z\right)dxdy$
là dạng bậc $2$ xác định trên một tập mở $U$ chứa $\varphi\left(D\right)$.
Theo định nghĩa 3.3, ta có:
\begin{alignat*}{1}
{\displaystyle \ensuremath{\int_{\varphi}{\alpha}}} & \coloneqq{\displaystyle \ensuremath{\int_{D}{\varphi^{*}\alpha}}}\\
 & =\int_{D}\left(\left(P\circ\varphi\right)dydz+\left(Q\circ\varphi\right)dzdx+\left(R\circ\varphi\right)dxdy\right)\\
 & =\int_{D}\left(\left(P,Q,R\right)\circ\varphi\right)\cdot\left(dydz,dzdx,dxdy\right).
\end{alignat*}
 Ta có:
\begin{alignat*}{1}
dydz & =\left(y_{u}du+y_{v}dv\right)\left(z_{u}du+z_{v}dv\right)\\
 & =y_{u}z_{v}dudv+y_{v}z_{u}dvdu\\
 & =\left(y_{u}z_{v}-y_{v}z_{u}\right)dudv,
\end{alignat*}
\begin{alignat*}{1}
dzdx & =\left(z_{u}du+z_{v}dv\right)\left(x_{u}du+x_{v}dv\right)\\
 & =z_{u}x_{v}dudv+z_{v}x_{u}dvdu\\
 & =\left(z_{u}x_{v}-z_{v}x_{u}\right)dudv,
\end{alignat*}
\begin{alignat*}{1}
dxdy & =\left(x_{u}du+x_{v}dv\right)\left(y_{u}du+y_{v}dv\right)\\
 & =x_{u}y_{v}dudv+x_{v}y_{u}dvdu\\
 & =\left(x_{u}y_{v}-x_{v}y_{u}\right)dudv,
\end{alignat*}
nên:
\begin{alignat*}{1}
 & \quad\;\left(dydz,dzdx,dxdy\right)\\
 & =\left(\left(y_{u}z_{v}-y_{v}z_{u}\right),\left(z_{u}x_{v}-z_{v}x_{u}\right),\left(x_{u}y_{v}-x_{v}y_{u}\right)\right)dudv\\
 & =\left|\begin{array}{ccc}
x_{u} & y_{u} & z_{u}\\
x_{v} & y_{v} & z_{v}
\end{array}\right|dudv\\
 & =\left(\varphi_{u}\times\varphi_{v}\right)dudv.
\end{alignat*}
Vậy
\begin{alignat*}{1}
{\displaystyle \ensuremath{\int_{\varphi}{\alpha}}} & ={\displaystyle \ensuremath{\int_{D}{\varphi^{*}\alpha}}}\\
 & ={\displaystyle \ensuremath{\int_{D}\left(\left(P,Q,R\right)\circ\varphi\right)\cdot\left(\varphi_{u}\times\varphi_{v}\right)dudv,}}
\end{alignat*}
điều này khớp với khái niệm tích phân mặt đã biết.\end{example}
\begin{defn}
Cho hai hình hộp $R=\left[a_{1},b_{1}\right]\times\left[a_{2},b_{2}\right]\cdots\left[a_{k},b_{k}\right]$
và $R'=\left[a_{1}',b_{1}'\right]\times\left[a_{2}',b_{2}'\right]\cdots\left[a_{k}',b_{k}'\right]$
trong $\mathbb{R}^{k}$. Ta gọi ánh xạ trơn $p:R'\rightarrow R$ là
một phép tham số hóa lại (reparametrization) nếu $p$ là song ánh
và ma trận Jacobi $J_{p}\left(s\right)$ của $p$ khả nghịch với mọi
$s\in R'$. Do $J_{p}\left(s\right)$ là ma trận vuông nên điều này
tương đương với $\text{det}J_{p}\left(s\right)\neq0$ với mọi $s\in R'$.
Do $p$ trơn nên $\text{det}J_{p}:R'\rightarrow\mathbb{R}$, $S\mapsto\text{det}J_{p}\left(s\right)$
là hàm liên tục, do đó hoặc $\text{det}J_{p}\left(s\right)>0$, hoặc
$\text{det}J_{p}\left(s\right)<0$ với mọi $s\in R'$. Nếu $\text{det}J_{p}\left(s\right)>0$
(tương ứng, $\text{det}J_{p}\left(s\right)<0$) ta nói $p$ bảo toàn
(tương ứng, đảo ngược) định hướng của $\varphi:R\rightarrow U$.
\end{defn}
Do $\text{det}J_{p}\left(s\right)\neq0$ với mọi $s\in R'$, bằng
định lý hàm ngược ta chứng minh được $p$ là một phép đổi biến, hay
một phép vi đồng phôi (diffeomorphism), tức $p$ là một song ánh,
khả vi liên tục và ánh xạ ngược $p^{-1}$ cũng khả vi liên tục. Từ
đó ta có định lý sau: 
\begin{thm}
Cho $\alpha$ là dạng bậc $k$ xác định trên tập mở $U$ và $\varphi:R\rightarrow U$
là hình hộp suy biến, $p:R'\rightarrow R$ là một phép tham số hóa
lại. Khi đó 
\[
{\displaystyle \ensuremath{\int_{\varphi\circ p}{\alpha}}}=\begin{cases}
\quad\int_{\varphi}\alpha & \text{nếu \ensuremath{p}bảo toàn định hướng,}\\
-\int_{\varphi}\alpha & \text{nếu \ensuremath{p}đảo ngược định hướng.}
\end{cases}
\]
\end{thm}
\begin{proof}
Ta có
\[
{\displaystyle \ensuremath{\int_{\varphi\circ p}{\alpha}}}={\displaystyle \ensuremath{\int_{R'}{\left(\varphi\circ p\right)^{*}\alpha}}}={\displaystyle \ensuremath{\int_{R'}{p^{*}\left(\varphi^{*}\alpha\right)}}}.
\]
Với $\varphi^{*}\alpha=gdx_{1}dx_{2}\cdots dx_{k}$ và $x=\left(x_{1},\ldots,x_{k}\right)=p\left(s\right)$,
$s\in R'$, từ ví dụ 2.3, ta có:
\begin{alignat*}{1}
p^{*}\left(\varphi^{*}\alpha\right) & =p^{*}\left(gdx_{1}dx_{2}\cdots dx_{k}\right)\\
 & =g\left(p\left(s\right)\right)\text{det}J_{p}\left(s\right)ds_{1}ds_{2}\cdots ds_{k},
\end{alignat*}
nên 
\[
{\displaystyle \ensuremath{\int_{\varphi\circ p}{\alpha}}}={\displaystyle \ensuremath{\int_{R'}g\left(p\left(s\right)\right)\text{det}J_{p}\left(s\right)ds_{1}ds_{2}\cdots ds_{k}}}.
\]
Do $p$ là một phép đổi biến, nên:
\begin{alignat*}{1}
 & \;\quad{\displaystyle \ensuremath{\int_{R'}g\left(p\left(s\right)\right)\text{det}J_{p}\left(s\right)ds_{1}ds_{2}\cdots ds_{k}}}\\
= & \begin{cases}
\quad\int_{R}g\left(x\right)dx_{1}dx_{2}\cdots dx_{k} & \text{nếu \ensuremath{\text{det}J_{p}\left(s\right)>0}},\\
-\int_{R}g\left(x\right)dx_{1}dx_{2}\cdots dx_{k} & \text{nếu \ensuremath{\text{det}J_{p}\left(s\right)<0}}.
\end{cases}
\end{alignat*}
Do $\int_{\varphi}\alpha$ chính là $\int_{R}g\left(x\right)dx_{1}dx_{2}\cdots dx_{k}$
nên ta có đpcm.\end{proof}
\begin{example}
Trong $\mathbb{R}^{k}$ cho khối vuông đơn vị $k$-chiều $\left[0,1\right]^{k}$
và một hình hộp $R=\left[a_{1},b_{1}\right]\times\left[a_{2},b_{2}\right]\cdots\left[a_{k},b_{k}\right]$.
Xét $p=\left(p_{1},p_{2},\ldots,p_{k}\right):\left[0,1\right]^{k}\rightarrow R$
với $p_{i,1\leq i\leq k}$ cho bởi $p_{i}\left(s\right)=\left(b_{i}-a_{i}\right)s_{i}+a_{i}$.
Dễ thấy $p:\left[0,1\right]^{k}\rightarrow R$ là song ánh. Ta có:
\[
J_{p}\left(s\right)=\left(\begin{array}{c}
\nabla p_{1}\\
\nabla p_{2}\\
\vdots\\
\nabla p_{k}
\end{array}\right)=\left(\begin{array}{cccc}
b_{1}-a_{1} & 0 & \cdots & 0\\
0 & b_{2}-a_{2} & \cdots & 0\\
\vdots & \vdots & \ddots & \vdots\\
0 & 0 & \cdots & b_{k}-a_{k}
\end{array}\right).
\]
Do đó $\text{det}J_{p}\left(s\right)=\prod_{i=1}^{k}\left(b_{i}-a_{i}\right)>0$,
nên $p$ bảo toàn định hướng. Vậy với $\alpha$ là dạng bậc $k$ trên
$U$ và $\varphi:R\rightarrow U$ thì $\int_{\varphi\circ p}\alpha=\int_{c}\alpha$.
\end{example}


Ví dụ 3.8 cho thấy rằng tích phân của dạng trên một hình hộp suy biến
$k$-chiều bất kỳ có thể quy về thành tích phân trên khối vuông (cube)
suy biến $\left[0,1\right]^{k}\rightarrow U$, ở phần tới ta chỉ xét
tích phân trên các khối vuông suy biến như vậy (điều này giúp giảm
đi sự rườm rà về ký hiệu).


\section{Định lý Stokes}

Định lý Stokes đã được giới thiệu trong các bài giảng về giải tích
cổ điển, với các trường hợp riêng như định lý Newton - Leibniz, định
lý Green... Nó nói rằng tích phân của một đối tượng $\alpha$ trên
biên $\partial D$ của một miền $D$ thì bằng tích phân trên $D$
của một đối tượng $d\alpha$ liên quan tới đạo hàm của $\alpha$,
tức là:
\[
{\displaystyle \int_{\partial D}\alpha=\int_{D}d\alpha.}
\]
Ở đây ta sẽ phát biểu và chứng minh định lý Stokes ở dạng tổng quát
hơn. Ở dạng đã biết, ta thấy $D$ và biên $\partial D$ không chỉ
là tập hợp, mà chúng còn được cho một định hướng, sao cho các định
hướng đó tương thích với nhau. Chẳng hạn ở định lý Newton - Leibniz:
\[
\int_{a}^{b}f'\left(x\right)dx=f\left(b\right)-f\left(a\right),
\]
với $\partial D=\left\{ a,b\right\} $ thì $\int_{\left\{ a,b\right\} }f=f\left(a\right)+f\left(b\right)$.
Như vậy với $\partial D$ đơn thuần chỉ mang nghĩa tập hợp thì không
đủ để phát biểu được định lý. Ở đây ta thấy vai trò của $a$ và $b$
không như nhau. Có thể nói rằng $a$ trước, $b$ sau, hoặc $a$ ứng
với dấu $-$, $b$ ứng với dấu $+$. Với sự phân biệt vai trò $a,b$
như vậy, đoạn $\left[a,b\right]$ một cách tương ứng cũng được gọi
là có định hướng từ $a$ đến $b$ (tích phân $\int_{a}^{b}f'\left(x\right)dx$
thực sự có ngầm ý một định hướng như vậy cho miền $\left[a,b\right]$).
Vậy tóm lại cần một định hướng cho biên, được cho bằng cách quy ước
các dấu $+,-$ cho các phần của nó.
\begin{defn}
(Mặt của khối vuông suy biến). Cho khối vuông suy biến $k$-chiều
$c:\left[0,1\right]^{k}\rightarrow U$, ta định nghĩa các mặt của
$c$, ký hiệu là $c_{i,0}$, $c_{i,1}$, như sau. Cho $t=\left(t_{1},t_{2},\ldots,t_{k-1}\right)\in\left[0,1\right]^{k-1}$,
với mọi $i=1,2,\ldots,k$, ta đặt
\begin{alignat*}{1}
c_{i,0}\left(t\right) & =c\left(t_{1},t_{2},\ldots,t_{i-1},0,t_{i},\ldots t_{k-1}\right),\\
c_{i,1}\left(t\right) & =c\left(t_{1},t_{2},\ldots,t_{i-1},1,t_{i},\ldots t_{k-1}\right).
\end{alignat*}
Chúng chính là giới hạn của $c$ lên các mặt của khối vuông đơn vị
$\left[0,1\right]^{k}$, là các khối vuông suy biến $\left(k-1\right)$-chiều.\end{defn}
\begin{example}
Cho $c:\left[0,1\right]\rightarrow U$ ($k=1$), $i$ chỉ bằng $1$,
ta có:
\begin{alignat*}{1}
c_{1,0}\left(t\right) & =c\left(0\right),\\
c_{1,1}\left(t\right) & =c\left(1\right).
\end{alignat*}


Cho $c:\left[0,1\right]^{2}\rightarrow U$ ($k=1$), $i=1,2$. Với
$t\in\left[0,1\right]$, ta có:
\begin{alignat*}{3}
c_{1,0}\left(t\right) & =c\left(0,t\right), & \qquad & c_{2,0}\left(t\right) & =c\left(t,0\right),\\
c_{1,1}\left(t\right) & =c\left(1,t\right), &  & c_{2,1}\left(t\right) & =c\left(t,1\right).
\end{alignat*}
\end{example}
\begin{defn}
(Biên của khối vuông suy biến). Cho khối vuông suy biến $c:\left[0,1\right]^{k}\rightarrow U$
với các mặt $c_{i,0}$, $c_{i,1}$, $i=1,2,\ldots,k$. Quy ước rằng
mặt $c_{i,0}$ mang dấu $\left(-1\right)^{i}$, mặt $c_{i,1}$ mang
dấu $\left(-1\right)^{i+1}$. Ta định nghĩa biên của $c$, ký hiệu
$\partial c$, là tổng hình thức (formal sum) của các mặt $c_{i,0}$,
$c_{i,1}$ cùng với dấu tương ứng, nghĩa là:
\[
\partial c=\sum_{i=1}^{k}\left(\left(-1\right)^{i}c_{i,0}+\left(-1\right)^{i+1}c_{i,1}\right)=\sum_{i=1}^{k}\left(-1\right)^{i}\left(c_{i,0}-c_{i,1}\right).
\]
\end{defn}
\begin{example}
Cho $c:\left[0,1\right]\rightarrow U$, ta có:
\begin{alignat*}{1}
\partial c & =\left(-1\right)\left(c_{1,0}-c_{1,1}\right)\\
 & =c_{1,1}-c_{1,0}.
\end{alignat*}


Cho $c:\left[0,1\right]^{2}\rightarrow U$, ta có:
\begin{alignat*}{1}
\partial c & =\sum_{i=1}^{2}\left(-1\right)^{i}\left(c_{i,0}-c_{i,1}\right)\\
 & =c_{1,1}-c_{1,0}+c_{2,0}-c_{2,1}.
\end{alignat*}

\end{example}
Các tổng hình thức trên, chẳng hạn $c_{1,1}-c_{1,0}+c_{2,0}-c_{2,1}$,
và cả các khối vuông suy biến $c$, là trường hợp riêng của khái niệm
xích dưới đây. 
\begin{defn}
(Xích). Cho $c_{1},c_{2},\ldots,c_{m}$ là một họ các khối vuông suy
biến $k$-chiều. Ta định nghĩa xích (chain) $c$ $k$-chiều là tổ
hợp tuyến tính (hình thức) của $c_{1},c_{2},\ldots,c_{m}$ với hệ
số thực, nghĩa là:
\[
c=a_{1}c_{1}+a_{2}b_{2}+\cdots+a_{m}b_{m},
\]
với $a_{1},a_{2},\ldots,a_{m}\in\mathbb{R}$. 
\end{defn}
Như vậy một khối vuông suy biến $c$ $k$-chiều cũng là một xích $k$-chiều,
biên $\partial c$ của nó là một xích $\left(k-1\right)$-chiều.
\begin{defn}
(Biên của xích). Biên của xích $c=a_{1}c_{1}+a_{2}b_{2}+\cdots+a_{m}b_{m}$,
ký hiệu $\partial c$, được định nghĩa là tổ hợp tuyến tính (hình
thức) của các biên $\partial c_{i}$ với hệ số $a_{i}$ tương ứng,
nghĩa là:
\[
\partial c=a_{1}\partial c_{1}+a_{2}\partial b_{2}+\cdots+a_{m}\partial b_{m}.
\]

\end{defn}
Với phép lấy tổ hợp (phép cộng) hình thức trên, dễ thấy rằng tập tất
cả các xích $k$-chiều là một không gian vectơ trên trường $\mathbb{R}$,
với cơ sở là tập tất cả các khối vuông suy biến $k$-chiều. Phép lấy
biên của xích như vậy là một ánh xạ (tuyến tính) từ không gian vectơ
các xích $k$-chiều đến không gian vectơ các xích $\left(k-1\right)$-chiều,
nó là mở rộng tuyến tính của phép lấy biên các khối vuông suy biến
$k$-chiều.

Bằng cách mở rộng tuyến tính - tương tự phép lấy biên của xích $c$
- khái niệm tích phân của dạng trên khối vuông suy biến lên cho xích,
ta định nghĩa khái niệm tích phân của dạng bậc $k$ trên xích như
sau.
\begin{defn}
(Tích phân của dạng trên xích). Cho $\alpha$ là dạng bậc $k$ xác
định trên tập mở $U$. Cho $c_{1},c_{2},\ldots,c_{m}$ là họ các khối
vuông suy biến $k$-chiều trong $U$ (tức $c_{i}\left(\left[0,1\right]^{k}\right)\subset U,$
$\forall i=1,2,\ldots,m$), xích $c=a_{1}c_{1}+a_{2}c_{2}+\cdots+a_{m}c_{m}$
được gọi là một xích $k$-chiều trong $U$. Tích phân của $\alpha$
trên $c$, ký hiệu $\int_{c}\alpha$, được cho bởi:
\[
\int_{c}\alpha=\sum_{i=1}^{m}a_{i}\int_{c_{i}}\alpha.
\]
\end{defn}
\begin{example}
Cho $\alpha$ là dạng bậc $1$ xác định trên tập mở $U$ và $c:\left[0,1\right]^{2}\rightarrow U$.
Từ ví dụ 3.13 ta có $\partial c=c_{1,1}-c_{1,0}+c_{2,0}-c_{2,1}$
và nó là một xích trong $U$. Theo định nghĩa trên, ta có:
\[
\int_{c}\alpha=\int_{c_{1,1}}\alpha-\int_{c_{1,0}}\alpha+\int_{c_{2,0}}\alpha-\int_{c_{2,1}}\alpha.
\]

\end{example}
Ta thấy rằng với các khái niệm trên, có thể phát biểu lại các dạng
của định lý Stokes đã biết. Chẳng hạn với định lý Newton - Leibniz
\[
\int_{a}^{b}f'\left(t\right)dt=f\left(b\right)-f\left(a\right),
\]
ta có thể viết như sau. Với $\varphi:\left[a,b\right]\rightarrow\mathbb{R}$
cho bởi $\varphi\left(t\right)=t$, ta có $\int_{a}^{b}f'\left(t\right)dt=\int_{\varphi}f'\left(x\right)dx$.
Bằng phép tham số hóa lại (ví dụ 3.8) $p:\left[0,1\right]\rightarrow\mathbb{R}$
cho bởi $p\left(s\right)=a+s\left(b-a\right)$, ta có $\int_{\varphi}f'\left(x\right)dx=\int_{c}f'\left(x\right)dx=\int_{c}df$,
$c=\varphi\circ p$ là một khối vuông suy biến. Mặt khác ta thấy $f\left(b\right)=f\left(c\left(1\right)\right)=\int_{\left\{ c\left(1\right)\right\} }f$,
tương tự $f\left(a\right)=\int_{\left\{ c\left(0\right)\right\} }f$,
nên $f\left(b\right)-f\left(a\right)=\int_{\left\{ c\left(1\right)\right\} }f-\int_{\left\{ c\left(0\right)\right\} }f=\int_{\left\{ c\left(1\right)\right\} -\left\{ c\left(0\right)\right\} }f=\int_{\partial c}f$.
Vậy $\int_{a}^{b}f'\left(t\right)dt=f\left(b\right)-f\left(a\right)$
được viết lại dưới dạng
\[
\int_{c}df=\int_{\partial c}f.
\]


Bây giờ ta phát biểu định lý Stokes.
\begin{thm}[Định lý Stokes]
 Cho $\alpha$ là dạng bậc $k-1$ xác định trên tập mở $U$ của $\mathbb{R}^{n}$
và $c$ là một xích $k$-chiều trong $U$. Khi đó:
\[
\boxed{\int_{c}d\alpha=\int_{\partial c}\alpha.}
\]
\end{thm}
\begin{proof}
Do phép lấy tích phân trên xích $c$ là mở rộng tuyến tính của phép
lấy tích phân trên khối vuông suy biến, nên ta chỉ cần chứng minh
định lý với $c$ là một khối vuông suy biến $\left[0,1\right]^{k}\rightarrow U$
là đủ.

Xét vế trái. Từ định nghĩa tích phân của dạng và định lý 2.4 (4),
ta có:
\[
\int_{c}d\alpha=\int_{\left[0,1\right]^{k}}c^{*}d\alpha=\int_{\left[0,1\right]^{k}}d\left(c^{*}\alpha\right).
\]
Do $\alpha$ là dạng bậc $k-1$ nên $c^{*}\alpha$ là dạng bậc $k-1$
trên $\left[0,1\right]^{k}$. Do đó ta có thể viết $c^{*}\alpha$
dưới dạng 
\[
c^{*}\alpha=\sum_{i=1}^{k}f_{i}dt_{1}dt_{2}\cdots\widehat{dt_{i}}\cdots dt_{k},
\]
với $f_{i}$ là các hàm thực trơn xác định trên $\left[0,1\right]^{k}$,
và $\widehat{dt_{i}}$ chỉ sự khuyết đi $dt_{i}$. Từ đó ta có: 
\begin{alignat}{1}
d\left(c^{*}\alpha\right) & =d\left(\sum_{i=1}^{k}f_{i}dt_{1}dt_{2}\cdots\widehat{dt_{i}}\cdots dt_{k}\right)\nonumber \\
 & =\sum_{i=1}^{k}d\left(f_{i}\right)dt_{1}dt_{2}\cdots\widehat{dt_{i}}\cdots dt_{k}\nonumber \\
 & =\sum_{i=1}^{k}\left(\sum_{j=1}^{k}\frac{\partial f_{i}}{\partial t_{j}}dt_{j}\right)dt_{1}dt_{2}\cdots\widehat{dt_{i}}\cdots dt_{k}.
\end{alignat}
Do với mọi $j\neq i$ thì $dt_{j}dt_{1}dt_{2}\cdots\widehat{dt_{i}}\cdots dt_{k}=0$
nên số hạng thứ $i$ trong $\left(3.2.1\right)$ là
\begin{alignat*}{1}
\left(\sum_{j=1}^{k}\frac{\partial f_{i}}{\partial t_{j}}dt_{j}\right)dt_{1}dt_{2}\cdots\widehat{dt_{i}}\cdots dt_{k} & =\frac{\partial f_{i}}{\partial t_{i}}dt_{i}dt_{1}dt_{2}\cdots\widehat{dt_{i}}\cdots dt_{k}\\
 & =\frac{\partial f_{i}}{\partial t_{i}}\left(-1\right)^{i+1}dt_{1}dt_{2}\cdots dt_{i}\cdots dt_{k}.
\end{alignat*}
Do đó
\[
d\left(c^{*}\alpha\right)=\left(\sum_{i=1}^{k}\left(-1\right)^{i+1}\frac{\partial f_{i}}{\partial t_{i}}\right)dt_{1}dt_{2}\cdots dt_{k}.
\]
Vậy vế trái bằng:
\begin{alignat}{1}
\int_{\left[0,1\right]^{k}}d\left(c^{*}\alpha\right) & =\int_{\left[0,1\right]^{k}}\left(\sum_{i=1}^{k}\left(-1\right)^{i+1}\frac{\partial f_{i}}{\partial t_{i}}\right)dt_{1}dt_{2}\cdots dt_{k}\nonumber \\
 & =\sum_{i=1}^{k}\left(-1\right)^{i+1}\int_{\left[0,1\right]^{k}}\frac{\partial f_{i}}{\partial t_{i}}dt_{1}dt_{2}\cdots dt_{k}.
\end{alignat}
Lưu ý lại rằng các tích phân ở vế phải của $\left(3.2.2\right)$ là
tích phân Riemann quen thuộc, nên áp dụng được định lý Fubini, và
do $f_{i,1\leq i\leq k}$ là các hàm trơn nên bằng định lý Newton
- Leibniz, ta có: 
\begin{alignat*}{1}
\int_{\left[0,1\right]^{k}}\frac{\partial f_{i}}{\partial t_{i}}dt_{1}dt_{2}\cdots dt_{k} & =\int_{\left[0,1\right]^{k-1}}\left(\int_{0}^{1}\frac{\partial f_{i}}{\partial t_{i}}dt_{i}\right)dt_{1}dt_{2}\cdots\widehat{dt_{i}}\cdots dt_{k}\\
 & =\int_{\left[0,1\right]^{k-1}}\left(f_{i}\left(t_{1},\ldots,t_{i-1},1,t_{i+1},\ldots,t_{k}\right)-f_{i}\left(t_{1},\ldots,t_{i-1},0,t_{i+1},\ldots,t_{k}\right)\right)dt_{1}dt_{2}\cdots\widehat{dt_{i}}\cdots dt_{k}.
\end{alignat*}


Xét vế phải. Với $\partial c=\sum_{i=1}^{k}\left(-1\right)^{i}\left(c_{i,0}-c_{i,1}\right)$,
ta có:
\[
\int_{\partial c}\alpha=\sum_{i=1}^{k}\left(-1\right)^{i}\left(\int_{c_{i,0}}\alpha-\int_{c_{i,1}}\alpha\right).
\]
Để so sánh với $\left(3.2.2\right)$ ta viết lại thành:
\[
\int_{\partial c}\alpha=\sum_{i=1}^{k}\left(-1\right)^{i+1}\left(\int_{c_{i,1}}\alpha-\int_{c_{i,0}}\alpha\right).
\]
Ta kiểm tra rằng 
\[
\int_{c_{i,0}}\alpha=\int_{\left[0,1\right]^{k-1}}\left(f_{i}\left(t_{1},\ldots,t_{i-1},0,t_{i+1},\ldots,t_{k}\right)\right)dt_{1}dt_{2}\cdots\widehat{dt_{i}}\cdots dt_{k}.
\]
Ta có:
\[
\int_{c_{i,0}}\alpha=\int_{\left[0,1\right]^{k-1}}c_{i,0}^{*}\alpha.
\]
Bằng cách viết $c_{i,0}:\left[0,1\right]^{k-1}\rightarrow U$, $\left(s_{1},s_{2},\ldots s_{k-1}\right)\mapsto c\left(s_{1},s_{2},\ldots,s_{i-1},0,s_{i},\ldots s_{k-1}\right)$
thành $c_{i,0}=c\circ h$, với $h:\left(s_{1},s_{2},\ldots,s_{i},\ldots s_{k-1}\right)\mapsto\left(s_{1},s_{2},\ldots,s_{i-1},0,s_{i},\ldots s_{k-1}\right)$,
ta có:
\begin{alignat*}{1}
c_{i,0}^{*}\alpha & =\left(c\circ h\right)^{*}\alpha\\
 & =h^{*}\left(c^{*}\alpha\right)\\
 & =h^{*}\left(\sum_{i=1}^{k}f_{i}dt_{1}dt_{2}\cdots\widehat{dt_{i}}\cdots dt_{k}\right).
\end{alignat*}
Với $j\neq i$, không mất tính tổng quát, giả sử $j<i$, ta có:
\begin{alignat*}{1}
h^{*}\left(f_{j}dt_{1}dt_{2}\cdots\widehat{dt_{j}}\cdots dt_{k}\right) & =\left(f_{j}\circ h\right)ds_{1}ds_{2}\cdots\widehat{ds_{j}}\cdots ds_{i-1}0ds_{i}\cdots ds_{k-1}\\
 & =0.
\end{alignat*}
Với $j=i$:
\begin{alignat*}{1}
h^{*}\left(f_{i}dt_{1}dt_{2}\cdots\widehat{dt_{i}}\cdots dt_{k}\right) & =\left(f_{i}\circ h\right)ds_{1}ds_{2}\cdots ds_{i-1}\widehat{0}ds_{i}\cdots dt_{k-1}\\
 & =\left(f_{i}\circ h\right)ds_{1}ds_{2}\cdots ds_{k-1}.
\end{alignat*}
Vậy nên
\begin{align*}
c_{i,0}^{*}\alpha & =h^{*}\left(\sum_{i=1}^{k}f_{i}dt_{1}dt_{2}\cdots\widehat{dt_{i}}\cdots dt_{k}\right)\\
 & =\left(f_{i}\circ h\right)ds_{1}ds_{2}\cdots ds_{k-1}.
\end{align*}
Vậy ta có:
\begin{align*}
\int_{c_{i,0}}\alpha & =\int_{\left[0,1\right]^{k-1}}c_{i,0}^{*}\alpha\\
 & =\int_{\left[0,1\right]^{k-1}}\left(f_{i}\circ h\right)ds_{1}ds_{2}\cdots ds_{k-1}\\
 & =\int_{\left[0,1\right]^{k-1}}f\left(s_{1},s_{2},\ldots,s_{i-1},0,s_{i},\ldots s_{k-1}\right)ds_{1}ds_{2}\cdots ds_{k-1}\\
 & =\int_{\left[0,1\right]^{k-1}}f_{i}\left(t_{1},\ldots,t_{i-1},0,t_{i+1},\ldots,t_{k}\right)dt_{1}dt_{2}\cdots\widehat{dt_{i}}\cdots dt_{k}.
\end{align*}
Hoàn toàn tương tự, ta có:
\[
\int_{c_{i,1}}\alpha=\int_{\left[0,1\right]^{k-1}}\left(f_{i}\left(t_{1},\ldots,t_{i-1},1,t_{i+1},\ldots,t_{k}\right)\right)dt_{1}dt_{2}\cdots\widehat{dt_{i}}\cdots dt_{k}.
\]
Vậy định lý đã được chứng minh.
\end{proof}

\chapter{Tích phân trên đa tạp }


\subsection*{Bài tập }
\begin{problem}
Bài tập 5.6 trang 65 giáo trình.
\end{problem}

\begin{problem}
Cho một $2$-hình hộp suy biến 
\begin{eqnarray*}
c:[0,1]^{2} & \to & \mathbb{R}^{4}\\
(t_{1},t_{2}) & \mapsto & (t_{1}^{2},t_{2},t_{1}t_{2},t_{1}+1).
\end{eqnarray*}
 Cho dạng $\alpha=x_{1}dx_{3}+x_{1}x_{4}dx_{2}$.
\begin{enumerate}
\item Tính cụ thể để kiểm tra rằng $c^{*}(d\alpha)=d(c^{*}\alpha)$.
\item Tính $\int_{c}d\alpha$. 
\item Tính $\int_{\partial c}\alpha$.
\item Kiểm tra $c$ là đơn ánh.
\item Chứng tỏ $c$ mang tập mở thành tập mở. Từ đó suy ra $c$ là đồng
phôi lên tập ảnh $c([0,1]^{2})$ của nó, tức ánh xạ ngược $c^{-1}$
là liên tục. (Gợi ý: chú ý $[0,1]^{2}$ là compắc và xét tập đóng
thay vì tập mở.) 
\item Tính đạo hàm $Dc$ và kiểm tra rằng tại mọi điểm $(t_{1},t_{2})\in(0,1)^{2}$
thì $Dc(t_{1},t_{2})$ là đơn ánh.
\item Chứng tỏ tập ảnh $S=c((0,1)^{2})$ là một đa tạp 2 chiều trong $\mathbb{R}^{4}$. 
\item Gọi $\mu$ là dạng thể tích trên $S$. Tính kéo lui $c^{*}\mu$.
\item Thiết lập công thức tích phân bội để tính diện tích của $S$. Hãy
ước lượng diện tích này.
\end{enumerate}
\end{problem}

\begin{problem}
Cho $\Omega$ là một tập con mở bị chặn của không gian Euclid $\mathbb{R}^{n}$.
Chứng tỏ $\Omega$ là một đa tạp $n$-chiều trong $\mathbb{R}^{n}$.
\end{problem}

\begin{problem}
Giả sử biên tôpô $\partial\Omega$ là một đa tạp $(n-1)$-chiều. Chứng
tỏ $\overline{\Omega}$ là một đa tạp $n$-chiều có biên là $\partial\Omega$. 
\end{problem}

\begin{problem}
Gọi $v$ là vectơ pháp tuyến đơn vị ngoài của $\partial\Omega$. Giả
sử trường vectơ $F$ trơn trên $\overline{\Omega}$. Viết $dx=\mu_{\Omega}$
và $dS=\mu_{\partial\Omega}$. Từ công thức Stokes hãy chứng tỏ:
\begin{equation}
\int_{\Omega}\text{div}F\ dx=\int_{\partial\Omega}F\cdot v\ dS.\label{thm:StokesPDE}
\end{equation}

\end{problem}

\begin{problem}
Viết $v=(v_{1},v_{2},\dotsc,v_{n})$. Giả sử hàm thực $f$ trơn trên
$\overline{\Omega}$. Từ ( \ref{thm:StokesPDE}) hãy chứng tỏ: 
\begin{equation}
\int_{\Omega}\frac{\partial f}{\partial x_{i}}\ dx=\int_{\partial\Omega}fv_{i}\ dS.\label{Stokescomp}
\end{equation}

\end{problem}

\begin{problem}
Giả sử hàm thực $g$ trơn trên $\overline{\Omega}$. Từ (\ref{Stokescomp})
hãy chứng minh \emph{công thức tích phân từng phần}: 
\begin{equation}
\int_{\Omega}\frac{\partial f}{\partial x_{i}}g\ dx=\int_{\partial\Omega}fgv_{i}\ dS-\int_{\Omega}f\frac{\partial g}{\partial x_{i}}\ dx.\label{eq:Green1}
\end{equation}

\end{problem}

\begin{problem}
Ta viết $\dfrac{\partial f}{\partial v}=\nabla f\cdot v$, đạo hàm
của $f$ theo hướng $v$. Nhắc lại toán tử Laplace $\Delta$ được
cho bởi $\Delta f=\sum_{i=1}^{n}\frac{\partial^{2}f}{\partial x_{i}^{2}}$.
Từ (\ref{Stokescomp}) hãy chứng minh \emph{công thức Green}:
\begin{equation}
\int_{\Omega}\Delta f\ dx=\int_{\partial\Omega}\frac{\partial f}{\partial v}\ dS.\label{eq:Green2}
\end{equation}

\end{problem}

\chapter{Bổ đề Poincaré }
\begin{thm}[bổ đề Poincaré]
 Mọi dạng trơn đóng trên đa tạp thắt được là khớp.
\end{thm}

\section{Chứng minh bổ đề Poincaré }

Giả sử $M\subset R^{n}$ là đa tạp thắt được. Xét tích $M\times\left[0;1\right]$,
gọi là mặt trụ với đáy $M$. Xét hai đồng luân $i_{0},i_{1}:M\rightarrow M\times\left[0,1\right]$
xác định bởi 
\[
i_{0}(x)=(x,0),
\]


\[
i_{1}(x)=(x,1).
\]
Phép đồng luân tương ứng từ $i_{0}$ tới $i_{1}$ là
\[
\begin{array}{cccc}
i: & \left[0;1\right]\times M & \rightarrow & M\times\left[0;1\right]\\
 & \left(t,x\right) & \rightarrow & \left(x;t\right).
\end{array}
\]
Ta có $i(x;0)=i_{0}(x)$; $i(x;1)=i_{1}(x)$. 

Tiếp theo ta xét toán tử:
\[
F:\:\Omega^{k+1}\left(M\times\left[0;1\right]\right)\rightarrow\Omega^{k}\left(M\right),\:k\in\left\{ 0;1;2;....;n\right\} ,
\]
với 
\[
\gamma=\underset{I}{\sum}f_{I}(x;t)dx_{I}+\underset{J}{\sum}g_{J}(x;t)dtdx_{J}\in\Omega^{k+1}
\]
 ($\gamma$ là dạng bậc $k+1$ , $I$ là bộ $k+1$ chỉ số, $J$ là
bộ $k$ chỉ số) thì 
\[
F\gamma=\underset{I}{\sum}\left(\intop_{0}^{1}g_{J}(x;t)dt\right)dx_{J}\in\Omega^{k}(M).
\]


Bây giờ ta chứng minh bổ đề Poincaré theo các bước sau.


\subsection*{Bước 1: }
\begin{prop}
$i_{1}^{*}\gamma-i_{0}^{*}\gamma=Fd\gamma+dF\gamma$ (hay $i_{1}^{*}-i_{0}^{*}=Fd+dF$).\end{prop}
\begin{proof}
Ta xét hai trường hợp.

Trường hợp $\gamma=f(x;t)dx_{I}$: Khi đó $F\gamma=0$ (do $\gamma$
không chứa $dt$) $\Rightarrow dF\gamma=0$. Trong khi đó
\[
d\gamma=\left(\dfrac{\partial f}{\partial t}dt\right)dx_{I}+\left(\underset{i=1}{\overset{n}{\sum}}\dfrac{\partial f}{\partial x_{i}}dx_{i}\right)dx_{I},
\]
 suy ra 
\[
Fd\gamma=\left(\underset{0}{\overset{1}{\intop}}\dfrac{\partial f}{\partial t}\left(x;t\right)dt\right)dx_{I}=\left[f\left(x;1\right)-f\left(x;0\right)\right]dx_{I}
\]


Ta lại có 
\begin{align*}
i_{1}^{*}\gamma-i_{0}^{*}\gamma & =i_{1}^{*}fdx_{I}+i_{0}^{*}fdx_{I}=\left[\left(f\circ i_{1}\right)-\left(f\circ i_{0}\right)\right]d\varphi_{I}\\
 & =\left[f\left(x;1\right)-f\left(x;0\right)\right]d\varphi_{i_{1}}d\varphi_{i_{2}}...d\varphi_{i_{k+1}}.
\end{align*}
Nhắc lại ở đây $M\overset{i_{1},i_{0}}{\longrightarrow}M\times\left[0;1\right]\overset{f}{\longrightarrow}\mathbb{R}$,
$x=\left(x_{1},x_{2},...,x_{n}\right)\in M$, $\left(x;t\right)=\left(x_{1},x_{2},...,x_{n},t\right)\in M\times\left[0;1\right]$,
$\varphi_{i}\left(x\right)=x_{i},\:i=1,2,...,n$. Đặt $\varphi_{n+1}^{1}\left(x\right)=1,\:\varphi_{n+1}^{0}\left(x\right)=0$,
ta có: 
\[
i_{1}\left(x\right)=\left(x;1\right)=\left(\varphi_{1}\left(x\right),\varphi_{2}\left(x\right),...,\varphi_{n}\left(x\right),\varphi_{n+1}^{1}\left(x\right)\right)
\]


\[
i_{0}\left(x\right)=\left(x;0\right)=\left(\varphi_{1}\left(x\right),\varphi_{2}\left(x\right),...,\varphi_{n}\left(x\right),\varphi_{n+1}^{0}\left(x\right)\right)
\]


\begin{align*}
d\varphi_{j_{1}}.d\varphi_{j_{2}}...d\varphi_{j_{k+1}} & =\left(\underset{i=1}{\overset{n}{\sum}}\dfrac{\partial\varphi_{j_{1}}}{\partial x_{i}}dx_{i}\right)\left(\underset{i=1}{\overset{n}{\sum}}\dfrac{\partial\varphi_{j_{2}}}{\partial x_{i}}dx_{i}\right)...\left(\underset{i=1}{\overset{n}{\sum}}\dfrac{\partial\varphi_{j_{k+1}}}{\partial x_{i}}dx_{i}\right)\\
 & =dx_{j_{1}}.dx_{j_{2}}...dx_{j_{k+1}}=dx_{I}.
\end{align*}
Suy ra $i_{1}^{*}\gamma-i_{0}^{*}\gamma=\left[f\left(x;1\right)-f\left(x;0\right)\right]dx_{I}$.
Vậy $i_{1}^{*}\gamma-i_{0}^{*}\gamma=Fd\gamma+dF\gamma$.

Trường hợp $\gamma=g\left(x;t\right)dtdx_{J}$: Ta có $F\gamma=\left(\underset{\mathrm{h\grave{a}m\ theo\ }x}{\underbrace{\intop_{0}^{1}g(x;t)dt}}\right)dx_{J}$,
suy ra 

\begin{align*}
dF\gamma & =\left(\underset{i=1}{\overset{n}{\sum}}\dfrac{\partial}{\partial x_{i}}\left(\intop_{0}^{1}g(x;t)dt\right)dx_{i}\right)dx_{J}=\left[\underset{i=1}{\overset{n}{\sum}}\left(\intop_{0}^{1}\dfrac{\partial g}{\partial x_{i}}(x;t)dt\right)dx_{i}\right]dx_{J}\\
 & =\underset{i=1}{\overset{n}{\sum}}\left(\intop_{0}^{1}\dfrac{\partial g}{\partial x_{i}}(x;t)dt\right)dx_{i}dx_{J}\\
 & =\underset{i=1}{\overset{n}{\sum}}\left(\intop_{0}^{1}\dfrac{\partial g}{\partial x_{i}}(x;t)dt\right)dx_{i}dx_{J}.
\end{align*}
 Mặt khác:

\begin{align*}
d\gamma & =d\left(gdtdx_{J}\right)=\left(\underset{i=1}{\overset{n}{\sum}}\dfrac{\partial g}{\partial x_{i}}(x;t)dx_{i}\right)dtdx_{J}\\
 & =\underset{i=1}{\overset{n}{\sum}}\left(\intop_{0}^{1}\dfrac{\partial g}{\partial x_{i}}(x;t)dt\right)dx_{i}dx_{J}\\
 & =\underset{i=1}{\overset{n}{\sum}}\dfrac{\partial g}{\partial x_{i}}(x;t)dx_{i}dtdx_{J}\\
 & =-\underset{i=1}{\overset{n}{\sum}}\dfrac{\partial g}{\partial x_{i}}(x;t)dtdx_{i}dx_{J}.
\end{align*}
 Suy ra 
\[
Fd\gamma=-\left[\underset{i=1}{\overset{n}{\sum}}\left(\intop_{0}^{1}\dfrac{\partial g}{\partial x_{i}}(x;t)dt\right)dx_{i}\right]dx_{J}=-\underset{i=1}{\overset{n}{\sum}}\left(\intop_{0}^{1}\dfrac{\partial g}{\partial x_{i}}(x;t)dt\right)dx_{i}dx_{J}.
\]
 Từ đó ta có $Fd\gamma+dF\gamma=0$. Mặt khác 

\begin{align*}
i_{1}^{*}\gamma-i_{0}^{*}\gamma & =i_{1}^{*}\left(gdtdx_{J}\right)-i_{0}^{*}\left(gdtdx_{J}\right)=\left(g\circ i_{1}\right)d\varphi_{n+1}^{1}d\varphi_{J}-\left(g\circ i_{0}\right)d\varphi_{n+1}^{0}d\varphi_{J}\\
 & =g\left(x;1\right)\left(\underset{i=1}{\overset{n}{\sum}}\dfrac{\partial1}{\partial x_{i}}dx_{i}\right)d\varphi_{J}-g\left(x;0\right)\left(\underset{i=1}{\overset{n}{\sum}}\dfrac{\partial0}{\partial x_{i}}dx_{i}\right)d\varphi_{J}=0.
\end{align*}
 Vậy $i_{1}^{*}\gamma-i_{0}^{*}\gamma=Fd\gamma+dF\gamma$ trong trường
hợp này.
\end{proof}

\subsection*{Bước 2:}
\begin{prop}[công thức đồng luân]
\index{công thức đồng luân} Cho các ánh xạ $\phi_{0},\phi_{1}:\,M\longrightarrow N$
($N$ là đa tạp) và $\phi:\,M\times\left[0;1\right]\longrightarrow N$
là phép đồng luân từ $\phi_{1}$ tới $\phi_{0}$. Khi đó với $\alpha$
là dạng bậc $k+1$ trên $N$, ta có: 
\[
\phi_{0}^{*}\alpha-\phi_{1}^{*}\alpha=F\phi^{*}d\alpha+dF\phi^{*}\alpha.
\]
\end{prop}
\begin{proof}
Xét sơ đồ $M$$\overset{i_{0},i_{1}}{\longrightarrow}M\times\left[0;1\right]\overset{\phi}{\longrightarrow}N$.
$\forall x\in M$ ta có:
\end{proof}
\[
\left(\phi\circ i_{0}\right)\left(x\right)=\phi\left(i_{0}\left(x\right)\right)=\phi\left(x,0\right)=\phi_{0}\left(x\right)
\]


\[
\left(\phi\circ i_{1}\right)\left(x\right)=\phi\left(i_{1}\left(x\right)\right)=\phi\left(x,1\right)=\phi_{1}\left(x\right).
\]
Suy ra $\phi_{0}=\phi\circ i_{0},\phi_{0}=\phi\circ i_{0},$ và 
\[
\begin{cases}
\phi_{0}*\alpha=\left(\phi\circ i_{0}\right)*\alpha=i_{0}*\phi*\alpha\\
\phi_{1}*\alpha=\left(\phi\circ i_{1}\right)*\alpha=i_{1}*\phi*\alpha.
\end{cases}
\]


Đặt $\gamma=\phi^{*}\alpha$ là dạng bậc $k+1$ trên $M\times\left[0;1\right]$.
Áp dụng kết quả ở bước 1, ta có:

\begin{eqnarray*}
\phi_{1}^{*}\alpha-\phi_{0}^{*}\alpha & = & i_{1}^{*}\gamma-i_{0}^{*}\gamma\\
 & = & Fd\gamma-dF\gamma\\
 & = & Fd\phi^{*}\alpha-dF\phi^{*}\alpha\\
 & = & F\phi^{*}d\alpha-dF\phi^{*}\alpha.
\end{eqnarray*}



\subsection*{Bước 3:}
\begin{prop}
Xét $\phi_{0},\phi_{1}:\,M\longrightarrow M$ và $\phi:\,M\times\left[0;1\right]\longrightarrow M$
là những phép đồng luân từ $\phi_{0}$ tới $\phi_{1}$. Đồng thời
$\phi$ là phép co rút vào $x_{0}\in M$, nghĩa là $\phi$ trơn, $\phi\left(x;0\right)=x_{0},\forall x\in M$
. Khi đó, với $\alpha$ là dạng bậc $k$ đóng trên $M$ với $k\geq1$
thì $\alpha$ khớp trên $M$, có $\beta=F\phi^{*}\alpha$ là dạng
bậc $k-1$ trên $M$ thoả mãn $d\beta=\alpha$. \end{prop}
\begin{proof}
Giả sử $\alpha=\underset{I}{\sum}f_{I}\left(x\right)d\varphi_{I}$,
ta có 
\begin{eqnarray*}
\phi_{1}^{*}\alpha & = & \left(\phi\circ i_{1}\right)^{*}\alpha\\
 & = & \underset{I}{\sum}\left[f_{I}\circ\left(\phi\circ i_{1}\right)\right]\left(x\right)d\varphi_{I}\\
 & = & \underset{I}{\sum}f_{I}\left[\phi\circ i_{1}\left(x\right)\right]d\varphi_{I}\\
 & = & \underset{I}{\sum}f_{I}\left[\phi\left(x;1\right)\right]d\varphi_{I}\\
 & = & \underset{I}{\sum}f_{I}\left(x\right)d\varphi_{i_{1}}.d\varphi_{i_{2}}...d\varphi_{i_{k}}\\
 & = & \underset{I}{\sum}\left[f_{I}\left(x\right)\overset{n}{.\underset{j=1}{\sum}}\dfrac{\partial\varphi_{i_{1}}}{\partial x_{j}}dx_{j}.\dfrac{\partial\varphi_{i_{2}}}{\partial x_{j}}dx_{j}...\dfrac{\partial\varphi_{i_{k}}}{\partial x_{j}}dx_{j}\right]\\
 & = & \underset{I}{\sum}f_{I}\left(x\right)dx_{i_{1}}.dx_{i_{2}}...dx_{i_{k}}\\
 & = & \underset{I}{\sum}f_{I}\left(x\right)d\varphi_{I}
\end{eqnarray*}
Vậy $\phi_{1}^{*}\alpha=\alpha$. Tương tự, ta có: 
\begin{eqnarray*}
\phi_{0}^{*}\alpha & = & \left(\phi\circ i_{0}\right)^{*}\alpha=\underset{I}{\sum}f_{I}\left[\phi\left(x;0\right)\right]d\varphi_{I}\\
 & = & \underset{I}{\sum}\left[f_{I}\circ\left(\phi\circ i_{0}\right)\right]\left(x\right)d\varphi_{I}\\
 & = & \underset{I}{\sum}f_{I}\left[\phi\circ i_{0}\left(x\right)\right]d\varphi_{I}\\
 & = & \underset{I}{\sum}f_{I}\left(x_{0}\right)d\varphi_{i_{1}}.d\varphi_{i_{2}}...d\varphi_{i_{k}}\\
 & = & 0.
\end{eqnarray*}
Bây giờ, đặt $\beta=F\phi^{*}\alpha$ và áp dụng kết quả ở bước 2,
ta có:
\[
\begin{array}{ccccc}
 & \phi_{1}^{*}\alpha-\phi_{0}^{*}\alpha & = & F\phi^{*}d\alpha+dF\phi^{*}\alpha\\
\Leftrightarrow & \alpha-0 & = & 0+d\beta\\
\Leftrightarrow & \alpha & = & d\beta.
\end{array}
\]
Vậy, với $\alpha$ là dạng bậc $k\geq1$, đóng thì $\alpha$ khớp.
Do đó, nếu $\alpha$ là dạng đóng trên đa tạp co rút được thì $\alpha$
khớp và bổ đề Poincaré được chứng minh xong.
\end{proof}

\section{Ví dụ }
\begin{example}
Trên $\mathbb{R}^{n}$, cho $\alpha=\underset{i=1}{\overset{n}{\sum}}f_{i}\left(x\right)dx_{i}$
là dạng bậc một đóng. Xét toán tử $F:\Omega^{k+1}\left(\mathbb{R}^{n}\times\left[0;1\right]\right)\longrightarrow\Omega^{k}\left(\mathbb{R}^{n}\right)$
xác định như sau: Với $\gamma=\underset{I}{\sum}f_{I}\left(x;t\right)dx_{I}+\underset{J}{\sum}g_{I}\left(x;t\right)dtdx_{J}\in\Omega$
thì $F\gamma=\underset{J}{\sum}\left[\int_{0}^{1}g_{I}\left(x;t\right)dt\right]dx_{J}$

Xét $\phi:R^{n}\times\left[0;1\right]\longrightarrow R^{n}$ xác định
như sau: $\phi\left(x;t\right)=tx,$ ta có $\phi\left(x;0\right)=$$O_{R^{n}},$$\phi\left(x;1\right)=x$.
Đặt $\beta=F\phi*\alpha$ thì theo bổ đề Poincare', ta có $\alpha=d\beta$
, nghĩa là $\alpha$ khớp.

Ta tìm hàm $\beta$ như sau. Ta có:
\begin{eqnarray*}
\phi*\alpha & = & \overset{n}{\underset{i=1}{\sum}}\left(f_{i}\circ\phi\right)d\varphi_{i}\\
 & = & \overset{n}{\underset{i=1}{\sum}}f_{i}\left[\phi\left(tx\right)\right]d\varphi_{i}\\
 & = & \overset{n}{\underset{i=1}{\sum}}f_{i}\left(tx\right)d\left(tx_{i}\right)\\
 & = & \overset{n}{\underset{i=1}{\sum}}f_{i}\left(tx\right)\left(x_{i}dt+tdx_{i}\right)\\
 & = & \overset{n}{\underset{i=1}{\sum}}f_{i}\left(tx\right)x_{i}dt+\overset{n}{\underset{i=1}{\sum}}f_{i}\left(tx\right)tdx_{i}
\end{eqnarray*}


\begin{eqnarray*}
\beta & = & F\phi*\alpha\\
 & = & \overset{n}{\underset{i=1}{\sum}}\left[\int_{0}^{1}f_{i}\left(tx\right)x_{i}dt\right]\\
 & = & \overset{n}{\underset{i=1}{\sum}}x_{i}\int_{0}^{1}f_{i}\left(tx\right)dt
\end{eqnarray*}


Bây giờ, áp dụng trên $R^{2}$, với $\alpha=ydx+xdy$ là dạng bậc
một đóng. Khi đó: 
\begin{eqnarray*}
\phi\left(x,y,t\right) & = & t\left(x,y\right)=\left(tx,ty\right)\\
\phi*\alpha & = & \left(ty.xdt+tx.ydt\right)+\left(ty.tdx+tx.tdy\right)\\
 & = & 2xytdt+t^{2}ydx+t^{2}xdy\\
\\
\beta & = & F\phi*\alpha\\
 & = & \int_{0}^{1}2xytdt\\
 & = & xy
\end{eqnarray*}


Ta có ngay $d\beta=\alpha$ , nghĩa là $\alpha$ khớp.\end{example}
\begin{thebibliography}{Sja06}
\bibitem[Sja06]{Sjamaar06} Reyer Sjamaar, \textit{Manifolds and differential
forms}, 2006, Cornell University. 

\end{thebibliography}
\printindex{}
\end{document}
